\documentclass[12pt,xcolor={dvipsnames},hyperref={pdftex,pdfpagemode=UseNone,hidelinks,pdfdisplaydoctitle=true},usepdftitle=false]{beamer}
\usepackage{presentation,math}
\def\pdf{xfairness.pdf}
\hypersetup{pdftitle={Lecture | Pricing under Fairness Concerns}}

\begin{document}

\title{Pricing under Fairness Concerns}
\information{Erik Eyster, Kristof Madarasz, Pascal Michaillat}%
{Journal of the European Economic Association, 2021}%
{https://www.pascalmichaillat.org/8.html}
\frame[plain]{\titlepage}

\begin{frame}
\frametitle{customers \& firms care about fair prices}
\begin{itemize}
\item evidence from marketing, psychology, sociology, economics
\item but pricing models never invoke fairness
\item[\then] pricing models do not have realistic microfoundations
\begin{itemize}
\item particularly problematic as these models are used for policy
\item example: Calvo pricing \& monetary policy
\end{itemize}
\item exception: theory by Rotemberg [2005]
\begin{itemize}
\item but somewhat difficult to analyze \& port to other models
\end{itemize}
\end{itemize}
\end{frame}

\begin{frame}
\frametitle{this paper: tractable theory of fair pricing}
\begin{itemize}
\item firms set prices to maximize profits given that
\begin{itemize}
	\item customers care about the fairness of markups
	\item customers systematically misperceive markups
\end{itemize}
\item in monopoly model:
	\begin{itemize}
		\item price rigidity (incomplete passthrough of costs into prices)
	\end{itemize}
\item in New Keynesian model:
\begin{itemize}
	\item short-run \& long-run nonneutrality of monetary policy 
\end{itemize}
\end{itemize}
\end{frame}

\begin{frame}
\heading{evidence that fairness matters}
\end{frame}

\begin{frame}
\frametitle{firms attribute price rigidity to fairness}
\begin{itemize}
\item 12,000 firms in the US, Canada, Europe, Japan say that they ``tacitly agree to stabilize prices, perhaps \al{out of fairness to customers}''
\begin{itemize}
\item Blinder et al [1998], Fabiani et al [2005], etc.
\end{itemize}
\item median rank of macro theories of price rigidity:
\begin{itemize}
	\item nominal contracts: 3/11
	\item menu costs: 9/11
	\item informational frictions: 11/11
\end{itemize}
\end{itemize}
\end{frame}

\begin{frame}
\frametitle{higher price due to higher markup is unfair}
\begin{itemize}
\item Kahneman, Knetsch, Thaler [1986]: ``A hardware store has been selling snow shovels for \$15. The morning after a large snowstorm, the store raises the price to \$20.''
\begin{itemize}
\item acceptable: 18\% 
\item \al{unfair: 82\%}
\end{itemize}
\end{itemize}
\end{frame}

\begin{frame}
\frametitle{but higher price with same markup is fair} 
\begin{itemize}
\item Kahneman, Knetsch, Thaler [1986]: ``Due to a transportation mixup, the wholesale price of lettuce has increased. A grocer has bought lettuce at a price that is 30 cents per head higher than normal. The grocer raises the price of lettuce to customers by 30 cents per head.''
\begin{itemize}
\item \al{acceptable: 79\%}
\item unfair: 21\%
\end{itemize}
\end{itemize}
\end{frame}

\begin{frame}
\frametitle{firms understand norms of fairness}
\begin{itemize}
\item Blinder et al [1998] surveyed 300 firms in the US
\item  64\% of firms: ``customers do not tolerate price increases after increases in demand'' 
\item 71\% of firms: ``customers do tolerate price increases after increases in cost''
\end{itemize}
\end{frame}

\begin{frame}
\frametitle{even God cares about markups}
\begin{itemize}
\item Talmudic law: maximum markup allowable in trade = 20\%
\item legal texts also regulate markups:
\begin{itemize}
	\item price of bread in France, 1700 -- 1970
	\item public utilities in the US
	\item anti-price-gouging legislation in most US states
\end{itemize}
\end{itemize}
\end{frame}

\begin{frame}
\frametitle{money illusion suggests misinference}
\begin{itemize}
\item Shafir, Diamond, Tversky [1997]: ``Imagine that within a six-month period all salaries and all prices went up by 25\%. You now earn and spend 25\% more than before. Six months ago, you were planning to buy a leather armchair whose price during the 6-month period went up from \$400 to \$500. Would you be more or less likely to buy the armchair now?''
\begin{itemize}
\item as or more likely: 62\%
\item \al{less likely: 38\%}
\end{itemize}
\end{itemize} 
\end{frame}

\begin{frame}
\heading{monopoly model\\with fairness concerns}
\end{frame}

\begin{frame}
\frametitle{customers}
\begin{itemize}
\item given price of consumption $P$, wealth $W$, and \al{fairness function $F$}
\item choose money balances $B$ and consumption $Y$	
\item to maximize quasilinear utility
\begin{equation*}
\frac{\e}{\e-1}  \bp{\al{F} \cdot Y}^{(\e-1)/\e}+B
\end{equation*}
\item subject to budget constraint $B + P \cdot Y = W$
\item different from social-preference approach to fairness
\begin{itemize}
	\item Rabin [1993] \then Rotemberg [2005]
\end{itemize}
\end{itemize}
\end{frame}

\begin{frame}
\frametitle{fairness function $F$}
\begin{itemize}
\item argument: perceived markup $M^p = P/C^{p}$
\begin{itemize}
\item $P$: observed price
\item $C^{p}$: perception of hidden marginal cost
\end{itemize}
\item positive: $F(M^p)>0$
\item decreasing: $F'(M^p)<0$
\begin{itemize}
\item higher markups are less fair
\end{itemize}
\item linear or concave: $F''(M^p) \leq 0$ 
\begin{itemize}
\item stronger response to increases in price than decreases
\end{itemize}
\end{itemize}
\end{frame}

\begin{frame}
\frametitle{examples of fairness function}
\includegraphics<1>[scale=\sfig,page=1]{\pdf}%
\includegraphics<2>[scale=\sfig,page=2]{\pdf}%
\end{frame}

\begin{frame}
\frametitle{perceived marginal cost}
\begin{equation*}
C^{p}(P)= (C^b)^{\g} \cdot \bs{\frac{P}{\e/(\e-1)}}^{1-\g}
\end{equation*}
\begin{itemize}
\item $C^b$: prior belief about monopoly's marginal cost
\item $P/[\e/(\e-1)]$: marginal cost with rational customers
\item $\g\in (0,1]$: amount of misinference
\begin{itemize}
\item $\g = 0$: rational inference
\item $0 < \g < 1$: some inference, but less than rational
\item $\g = 1$: no inference
\end{itemize}
\end{itemize}
\end{frame}

\begin{frame}
\frametitle{perceived markup}
\begin{equation*}
M^p(P) = \frac{P}{C^{p}(P)} = \bp{\frac{\e}{\e-1}}^{1-\g}  \bp{\frac{P}{C^b}}^{\g}
\end{equation*}
\begin{itemize}
\item misinference ($\g>0$): \al{$M^p$ increasing in $P$}
\begin{itemize}
\item when a price rises due to a cost increase, customers partially misattribute the higher price to a higher markup
\end{itemize}
\item rational inference ($\g = 0$): constant $M^p$
\begin{itemize}
\item when a price rises due to a cost increase, customers realize that the profit-maximizing markup is constant
\end{itemize}
\end{itemize}
\end{frame}

\begin{frame}
\frametitle{demand curve}
\begin{equation*}
Y^{d}(P)= P^{-\e}  \cdot  \al{F(M^p(P))^{\e-1}}
\end{equation*}
\begin{itemize}
\item $P^{-\e}$: traditional effect of price on demand
\begin{itemize}
 \item price \then customers' budget sets \then demand
 \end{itemize}
\item \al{$F(M^p(P))^{\e-1}$: effect of price on demand through fairness}
\begin{itemize}
\item price \then perceived markup \then perceived fairness \then marginal utility of consumption \then demand
\end{itemize}
\end{itemize}
\end{frame}

\begin{frame}
\frametitle{monopoly}
\begin{itemize}
\item given marginal cost of production $C$
\begin{itemize}
	\item unobservable to customers
\end{itemize}
\item chooses output $Y$ and price $P$
\item to maximize profits $Y \cdot (P-C)$
\item subject to customers' demand $Y = Y^d(P)$
\end{itemize}
\end{frame}


\begin{frame}
\frametitle{profit-maximizing price}
\begin{itemize}    
\item profit-maximizing price: 
\begin{equation*}
P = M \cdot C
\end{equation*}
\item $M$: profit-maximizing markup
\begin{equation*}
M=\frac{E}{E-1}
\end{equation*}
\item $E$: (positive) elasticity of demand wrt price
\begin{equation*}
E = - \frac{P}{Y^d} \cdot \od{Y^{d}}{P} 
\end{equation*}
\end{itemize}
\end{frame}

\begin{frame}
\frametitle{price elasticity of demand}
\begin{itemize}
\item $Y^d(P) = P^{-\e}  \cdot F(M^p(P))^{\e-1}$
\item price elasticity of perceived markup $=\g$
\item $\f(M^p) = $ (positive) elasticity of fairness function wrt markup
\item then we obtain:
\begin{equation*}
E(P) = \e+ \al{(\e-1)  \cdot  \g \cdot \f(M^p(P))}
\end{equation*}
\item fairness operates through term $\al{(\e-1)  \cdot  \g \cdot \f(M^p(P))}$ in price elasticity of demand
\end{itemize}
\end{frame}

\begin{frame}
\frametitle{elasticity of fairness function wrt markup}
\begin{equation*}
\f(M^p) = - \frac{M^p}{F(M^p)}\cdot \od{F}{M^p}
\end{equation*}
\begin{itemize}
\item $\f>0$
\begin{itemize}
\item because $F>0$
\item and $F'<0$
\end{itemize}
\item \al{$\f$ increasing in $M^p$}
\begin{itemize}
\item because $F$ is decreasing in $M^p$
\item and $-F'$ is weakly increasing in $M^p$ (concavity of $F$)
\end{itemize}
\end{itemize}
\end{frame}

\begin{frame}
\frametitle{no fairness concerns \then flexible prices}
\begin{equation*}
E(P) = \e+ (\e-1) \cdot  \g \cdot \al{\underset{=\;0}{\f(M^p(P))}}
\end{equation*}
\begin{itemize}
\item standard price elasticity of demand: $E=\e$
\item standard markup: $M=\e/(\e-1)$
\item passthrough of marginal costs into prices $= 100\%$
\begin{itemize}
\item because markup is constant
\end{itemize}
\end{itemize}
\end{frame}

\begin{frame}
\frametitle{rational inference \then flexible prices}
\begin{equation*}
E(P) = \e+ (\e-1)\cdot \al{\underset{=\;0}{\g}}\cdot \f(M^p(P))
\end{equation*}
\begin{itemize}
\item standard price elasticity of demand: $E=\e$
\item standard markup: $M=\e/(\e-1)$
\item marginal-cost passthrough $= 100\%$
\begin{itemize}
\item because markup is constant
\end{itemize}
\end{itemize}
\end{frame}

\begin{frame}
\frametitle{fairness \& misinference \then more competition}
\begin{equation*}
E(P) = \e+ (\e-1)\cdot \al{\underset{>0}{\g}} \cdot \al{\underset{>0}{\f(M^p(P))}}
\end{equation*}
\begin{itemize}
\item price elasticity of demand is higher: $E>\e$
\item markup is lower:
\begin{equation*}
M = \frac{E}{E-1} < \frac{\e}{\e-1}
\end{equation*}
\end{itemize}
\end{frame}

\begin{frame}
\frametitle{fairness \& misinference \then price rigidity}
\begin{itemize}
\item equilibrium markup is a fixed point:
\begin{equation*}
M = \frac{E(M \cdot C)}{E(M \cdot C)-1}
\end{equation*}
\item  equilibrium markup satisfies
\begin{equation*}
M=1+\frac{1}{\e-1}\cdot \frac{1}{1+\al{\g \cdot\f(M^p(M \cdot C))}}
\end{equation*}
\item[\then] marginal-cost passthrough $< 100\%$
\begin{itemize}
\item because markup \down when marginal cost \up
\end{itemize}
\end{itemize}
\end{frame}

\begin{frame}
\frametitle{evidence of incomplete passthrough}
\begin{itemize}
\item labor-cost shocks in Sweden: passthrough = 30\%
\begin{itemize}
	\item Carlsson, Skans [2012]
\end{itemize}
\item reduction in import tariff in India: passthrough = 30\%--40\%
\begin{itemize}
	\item De Loecker et al [2016]
\end{itemize}
\item marginal-cost shocks in Mexico: passthrough = 20\%--40\%
\begin{itemize}
	\item Caselli, Chatterjee, Woodland [2017] 
\end{itemize}
\item energy-price shocks in the US: passthrough = 50\%--70\%
\begin{itemize}
	\item Ganapati, Shapiro, Walker [2020] 
\end{itemize}
\end{itemize}
\end{frame}


\begin{frame}
\heading{new keynesian model\\with fairness concerns}
\end{frame}

\begin{frame}
\frametitle{fairness concerns}
\begin{itemize}
\item fairness-adjusted consumption of good $i$ by household $j$: 
\begin{equation*}
Z_{ij}= \al{F_{i}(M^p_{i}(P_{i}))} \cdot Y_{ij}
\end{equation*}
\item fairness-adjusted consumption by household $j$ is aggregated:
\begin{equation*}
Z_{j} = \bs{\int_{0}^{1} Z_{ij}^{(\e-1)/\e}di}^{\e/(\e-1)}
\end{equation*}
\item consumption index $Z_j$ enters utility
\begin{equation*}
\E[0]{\sum \d^{t}  \bs{\ln{Z_{j}}-\frac{N_{j}(t)^{1+\eta}}{1+\eta}}}
\end{equation*}
\end{itemize}
\end{frame}


\begin{frame}
\frametitle{misinference}
\begin{itemize}
\item endogenize parameter $C^b$ using past belief
\item perceived marginal cost of good $i$ in period $t$:
\begin{equation*}
C^{p}_{i}(t)= \bs{\al{C^{p}_{i}(t-1)}}^{\g} \cdot \bs{\frac{P_{i}(t)}{\e/(\e-1)}}^{1-\g}
\end{equation*}
\item $\g\in(0,1]$: misinference
\end{itemize}
\end{frame}

\begin{frame}
\frametitle{Short-run monetary nonneutrality}
\begin{itemize}
\item 3 equilibrium variables: $\wh{m^p}(t)$, $\wh{n}(t)$, and $\wh{\pi}(t)$
\item belief dynamics: $\wh{m^p}(t) = \g \cdot \bs{\wh{\pi}(t) + \wh{m^p}(t-1)}$
\item IS equation:
\begin{equation*}
\a \wh{n}(t) +\psi \wh{\pi}(t)= \a \E[t]{\wh{n}(t+1)}+\E[t]{\wh{\pi}(t+1)} - s(t)
\end{equation*}    	
\item short-run Phillips curve
\begin{equation*}
(1-\d\g) \wh{m^p}(t) -\l_1 \wh{n}(t) = \d\g \E[t]{\wh{\pi}(t+1)} -\l_2 \E[t]{\wh{n}(t+1)}
\end{equation*}
\item nonneutrality arises from Phillips curve
\item evidence: Christiano, Eichenbaum, Evans [1999]; Ramey [2016]
\end{itemize}
\end{frame}

\begin{frame}
\frametitle{hybrid short-run phillips curve}
\begin{itemize}
\item Phillips curve is forward-looking + \al{backward-looking}
\begin{equation*}
(1-\d\g) \al{\sum_{s=0}^{+\infty}\g^{s+1} \wh{\pi}(t-s)} -\l_1 \wh{n}(t)  = \d\g \E[t]{\wh{\pi}(t+1)} -\l_2 \E[t]{\wh{n}(t+1)}
\end{equation*}
\item hybrid short-run Phillips curve is more realistic
\begin{itemize}
	\item inflation dynamics are more persistent
\end{itemize}
\item evidence: Mavroeidis, Plagborg-Moller, Stock [2014]
\end{itemize}
\end{frame}

\begin{frame}
\frametitle{calibration from passthrough evidence}
\includegraphics[scale=\sfig,page=3]{\pdf}%
\end{frame}

\begin{frame}
\frametitle{Loosening of monetary policy}
\includegraphics<1>[scale=\sfig,page=4]{\pdf}%
\includegraphics<2>[scale=\sfig,page=5]{\pdf}%
\includegraphics<3>[scale=\sfig,page=6]{\pdf}%
\includegraphics<4>[scale=\sfig,page=7]{\pdf}%
\includegraphics<5>[scale=\sfig,page=8]{\pdf}%
\includegraphics<6>[scale=\sfig,page=9]{\pdf}%
\end{frame}

\begin{frame}
\frametitle{explanation for anger at inflation}
\begin{itemize}
\item Shiller [1997] surveyed 120 people in the US
\item 85\% said that ``when they go to the store and see that prices are higher, they sometimes \al{feel a little angry} at someone''
\item someone:  ``\al{greedy} store owners and businesses''
\end{itemize}
\end{frame}

\begin{frame}
\frametitle{explanation for opinions about price movements in japan (BOJ survey, 2001--2017)}
\begin{table}
\begin{small}\begin{tabular*}{\textwidth}{@{\extracolsep\fill}lccc}
perceived price change	& favorable & neutral & unfavorable \\ 
\toprule 
prices have gone up & 2.5\% & 13.0\% & \al{83.7\%} \\ 
($N=68,491$)	&	& & \\ 
prices have gone down & \al{43.0\%}	& 34.2\% & 21.9\% \\ 
($N=18,257$)	& & & \\ 
\bottomrule\end{tabular*}\end{small}
\end{table}
\end{frame}

\begin{frame}
\frametitle{Improvement in technology}
\includegraphics<1>[scale=\sfig,page=10]{\pdf}%
\includegraphics<2>[scale=\sfig,page=11]{\pdf}%
\includegraphics<3>[scale=\sfig,page=12]{\pdf}%
\includegraphics<4>[scale=\sfig,page=13]{\pdf}%
\includegraphics<5>[scale=\sfig,page=14]{\pdf}%
\includegraphics<6>[scale=\sfig,page=15]{\pdf}%
\end{frame}

\begin{frame}
\frametitle{Long-run monetary nonneutrality}
\begin{itemize}
\item steady-state perceived markup:
\begin{equation*}
\ln{\ol{M^p}} = \ln{\frac{\e}{\e-1}} + \al{\frac{\g}{1-\g}} \cdot \ol{\pi}
\end{equation*}
\item higher inflation \then higher perceived markup \then lower fairness \then lower actual markup \then higher output
\item evidence of long-run nonneutrality: King, Watson [1994, 1997]
\item evidence on inflation \& markups: Benabou [1992]; Banerjee, Russell [2005]
\item nonneutrality modulated by acclimation to inflation: $\c\in[0,1]$
\end{itemize}
\end{frame}

\begin{frame}
\frametitle{long-run Phillips curve}
\includegraphics<1>[scale=\sfig,page=16]{\pdf}%
\includegraphics<2>[scale=\sfig,page=17]{\pdf}%
\end{frame}

\end{document}
