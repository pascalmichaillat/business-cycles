\documentclass[12pt,xcolor={dvipsnames},hyperref={pdftex,pdfpagemode=UseNone,hidelinks,pdfdisplaydoctitle=true},usepdftitle=false]{beamer}
\usepackage{presentation,math}
\def\pdf{xmultiplier.pdf}
\hypersetup{pdftitle={Lecture | A Theory of Countercyclical Government Multiplier}}

\begin{document}

\title{A Theory of Countercyclical Government Multiplier}
\information{Pascal Michaillat}%
{American Economic Journal: Macroeconomics, 2014}%
{https://www.pascalmichaillat.org/2.html}
\frame[plain]{\titlepage}

\begin{frame}
\frametitle{government multiplier is countercyclical}
\begin{itemize}
\item US evidence:
\begin{itemize}
\item Auerbach \& Gorodnichenko [2012]
\item Candelon \& Lieb [2013]
\item Fazzari, Morley, \& Panovska [2015]
\end{itemize}
\item international evidence:
\begin{itemize}
\item Auerbach \& Gorodnichenko [2013]
\item Jorda \& Taylor [2016]
\item Holden \& Sparrman [2018]
\end{itemize}
\end{itemize}
\end{frame}

\begin{frame}
\frametitle{existing explanation: zero lower bound}
\begin{itemize}
\item multiplier is large in bad times because of the zero lower bound
\begin{itemize}
\item Eggertsson [2011]
\item Christiano, Eichenbaum, \& Rebelo [2011] 
\item Eggertsson \& Krugman [2012]
\end{itemize}
\item but evidence of countercyclical multipliers is obtained away from the zero lower bound
\end{itemize}
\end{frame}

\begin{frame}
\frametitle{this paper's explanation: labor market slack}
\begin{itemize}
\item multiplier $\equiv$ additional number of employed workers when 1 worker is hired in the public sector 
\item multiplier doubles when unemployment rises from 5\% to 8\%
\begin{itemize}
\item irrespective of the zero lower bound
\end{itemize}
\item mechanism based on the matching model of the labor market from \href{https://www.pascalmichaillat.org/1.html}{Michaillat [2012]}
\begin{itemize}
\item unemployment = rationing + frictional
\end{itemize}
\end{itemize}
\end{frame}

\begin{frame}
\frametitle{importance of public employment}
\begin{itemize}
\item public employment $= 63\%$ of government consumption expenditures in the US, 1947--2011
\begin{itemize}
\item even more if purchase of services (contractors) are included
\end{itemize}
\item stimulus packages often raise public employment
\begin{itemize}
\item Great Depression [Neumann, Fishback, \& Kantor 2010]
\end{itemize} 
\end{itemize}
\end{frame}

\begin{frame}
\frametitle{mechanism: crowding out}
\begin{itemize}
\item public employment crowds out private employment
\begin{itemize}
\item because government and firms compete for the same jobseekers
\end{itemize}
\item formally: an increase in public employment raises labor market tightness
\begin{itemize}
\item[\then] raises recruiting costs
\item[\then] reduces private employment
\end{itemize}
\end{itemize}
\end{frame}

\begin{frame}
\frametitle{mechanism: bad times vs. good times}
\begin{itemize}
\item bad times: labor demand is low so unemployment is high and competition for workers is weak
\begin{itemize}
\item[\then] weak crowding out
\end{itemize}
\item good times: labor demand is high so unemployment is low and competition for workers is strong
\begin{itemize}
\item[\then] strong crowding out
\end{itemize}
\item procyclical crowding out \then countercyclical multiplier
\end{itemize}
\end{frame}

\begin{frame}
\heading{matching model\\with public employment}
\end{frame}

\begin{frame}
\frametitle{public employment}
\begin{itemize}
\item the government employs $g_{t}$ workers
\begin{itemize}
\item public employment is financed by an income tax
\end{itemize}
\item public and private jobs are identical
\begin{itemize}
\item same wage $w$
\item same job-separation rate $s$
\end{itemize}
\item unemployed workers indiscriminately apply to public and private jobs
\item public and private vacancies compete for the same unemployed workers 
\end{itemize}
\end{frame}

\begin{frame}
\frametitle{matching function}
\includegraphics<1>[scale=\sfig,page=1]{\pdf}%
\includegraphics<2>[scale=\sfig,page=2]{\pdf}%
\includegraphics<3>[scale=\sfig,page=3]{\pdf}%
\end{frame}


\begin{frame}
\frametitle{worker flows: job creation \& separation}
\includegraphics<1>[scale=\sfig,page=4]{\pdf}%
\includegraphics<2>[scale=\sfig,page=5]{\pdf}%
\includegraphics<3>[scale=\sfig,page=6]{\pdf}%
\end{frame}

\begin{frame}
\frametitle{labor supply}
\begin{itemize}
\item labor supply $\equiv$ workers' employment rate when labor market flows are balanced
\item balanced flows: E $\to$ U = U $\to$ E
\begin{itemize}
\item $s\cdot n = f(\t)\cdot u = f(\t)\cdot \bs{1-n+s\cdot n}$
\end{itemize}
\item expression for labor supply:
\begin{equation*}
n^{s}(\underset{+}{\t})=\frac{f(\t)}{s+ (1-s)\cdot  f(\t)}
\end{equation*}
\begin{itemize}
\item equivalent to the Beveridge curve
\end{itemize}
\end{itemize}
\end{frame}

\begin{frame}
\frametitle{representative firm}
\begin{itemize}
\item hires $l_{t}-(1-s)\cdot l_{t-1}$ new workers by posting vacancies
\begin{itemize}
\item cost per vacancy: $r\cdot a$
\item vacancy-filling probability: $q(\t_{t})$
\end{itemize}
\item employs $l_{t}$ workers paid $w$
\item production function:  $y_{t}= a\cdot l_{t}^{\a}$
\begin{itemize}
\item $a$: level of technology
\item $\a\in(0,1]$: marginal returns to labor
\end{itemize}
\end{itemize}
\end{frame}

\begin{frame}
\frametitle{firm's problem}
\begin{itemize}
\item given wage and tightness $\bc{w, \t_t}$, the firm chooses employment $\bc{l_{t}}$ to maximize discounted profits
\begin{equation*}
\sum_{t=0}^{+\infty} \b^t \cdot\bs{\underbrace{a \cdot l_t^{\a}}_{\text{production}}-\underbrace{w\cdot l_t}_{\text{wage bill}}-\underbrace{\frac{r\cdot a}{q(\t_{t})}}_{\text{hiring cost}}\cdot \underbrace{\bs{l_{t}-(1-s)\cdot l_{t-1}}}_{\text{new hires}}}
\end{equation*}
\end{itemize}
\end{frame}

\begin{frame}
\frametitle{private labor demand}
\begin{itemize}
\item first-order condition with respect to $l$ in steady state:
\begin{equation*}
\underbrace{a\cdot\a\cdot l^{\a-1}}_{\text{marginal product of labor}} = \underbrace{w}_{\text{wage}} +\underbrace{\bs{1-\b\cdot (1-s)}\cdot \frac{r\cdot a}{q(\t)}}_{\text{recruiting cost}}
\end{equation*}
\item given $\t$ and $w$, the private labor demand is firms' desired employment rate in steady state:
\begin{equation*}
l^{d}(\underset{-}{\t},\underset{-}{w}) = \bs{\frac{1}{\a}\cdot \bc{\frac{w}{a}+\bs{1-\b\cdot (1-s)}\cdot \frac{r}{q(\t)}}}^{\frac{-1}{1-\a}}
\end{equation*}
\end{itemize}
\end{frame}

\begin{frame}
\frametitle{wage schedule}
\begin{itemize}
\item there are mutual gains from matching
\item many wage schedules are consistent with equilibrium
\item we assume a simple wage schedule: $w= \o\cdot a^{\g}$
\begin{itemize}
\item $\g=0$: fixed wage (unresponsive to $a$)
\item $\g=1$: flexible wage (proportional to $a$) 
\item $\g\in(0,1)$: partially rigid wage (subproportional to $a$)
\end{itemize}
\end{itemize}
\end{frame}


\begin{frame}
\frametitle{aggregate labor demand}
\begin{itemize}
\item using the wage schedule, we rewrite the private labor demand as a function of $\t$ and $a$:
\begin{equation*}
l^{d}(\underset{-}{\t},\underset{+}{a}) = \bs{\frac{1}{\a}\cdot \bc{\o\cdot a^{\g-1}+\bs{1-\b\cdot (1-s)}\cdot \frac{r}{q(\t)}}}^{\frac{-1}{1-\a}}
\end{equation*}
\item aggregate labor demand:
\begin{equation*}
n^{d}(\underset{-}{\t},\underset{+}{a},\underset{+}{g}) = l^{d}(\t,a) + g
\end{equation*}
\end{itemize}
\end{frame}


\begin{frame}
\frametitle{steady-state equilibrium}
\begin{itemize}
\item tightness equalizes labor supply and demand:
\begin{equation*}
n^{s}(\underset{+}{\t})= n^{d}(\underset{-}{\t},\underset{+}{a},\underset{+}{g})
\end{equation*}
\item recession: low technology $a$
\item expansion: high technology $a$
\item stimulus: high public employment $g$
\item note: in matching models, the convergence to steady state is almost immediate [Hall 2005]
\end{itemize}
\end{frame}

\begin{frame}
\frametitle{equilibrium diagram}
\includegraphics[scale=\sfig,page=7]{\pdf}%
\end{frame}

\begin{frame}
\heading{properties of the multiplier}
\end{frame}

\begin{frame}
\frametitle{definition of the multiplier}
\begin{itemize}
\item the multiplier is $\l\equiv \odx{n}{g}$
\begin{itemize}
\item additional number of employed workers when 1 worker is hired in the public sector
\end{itemize}
\item another expression: $\l=1+\odx{l}{g}$
\begin{itemize}
\item $1$: mechanical effect of public employment
\item $\odx{l}{g}<0$: crowding out of private employment by public employment
\item weaker crowding out \so larger multiplier 
\end{itemize}
\end{itemize}
\end{frame}

\begin{frame}
\frametitle{assumptions from \href{https://www.pascalmichaillat.org/1.html}{Michaillat [2012]}}
\begin{itemize}
\item $\a<1$: diminishing marginal returns to labor in production
\begin{itemize}
\item[\then] in $(n,\t)$ plane: $n^{d}(\t,a,g)$ is downward-sloping 
\end{itemize}
\item $\g<1$: partial wage rigidity
\begin{itemize}
\item[\then] in $(n,\t)$ plane: $n^{d}(\t,a,g)$ shifts inward when $a$ rises
\end{itemize}
\end{itemize}
\end{frame}

\begin{frame}
\frametitle{multiplier properties when $\a<1$ and $\g<1$}
\begin{itemize}
\item multiplier $< 1$
\begin{itemize}
\item there is crowding out of private employment by public employment
\end{itemize}
\item but multiplier $> 0$
\begin{itemize}
\item crowding out is less than one-for-one
\end{itemize}
\item multiplier is larger when $a$ is lower
\begin{itemize}
\item higher unemployment \then weaker crowding out \then larger multiplier
\end{itemize}
\end{itemize}
\end{frame}

\begin{frame}
\frametitle{positive multiplier}
\includegraphics<1>[scale=\sfig,page=8]{\pdf}%
\includegraphics<2>[scale=\sfig,page=9]{\pdf}%
\includegraphics<3>[scale=\sfig,page=10]{\pdf}%
\includegraphics<4>[scale=\sfig,page=11]{\pdf}%
\end{frame}    

\begin{frame}
\frametitle{countercyclical multiplier}
\includegraphics<1>[scale=\sfig,page=11]{\pdf}%
\includegraphics<2>[scale=\sfig,page=12]{\pdf}%
\end{frame}

\begin{frame}
\frametitle{intuition for the mechanism}
\begin{itemize}
\item when unemployment is high:
\begin{itemize}
\item government hires unemployed workers who would not have been hired otherwise
\item[\then] public employment does not affect private employment much
\end{itemize}
\item but when unemployment is low:
\begin{itemize}
\item government hires workers that would have been hired by the private sector otherwise
\item[\then] public employment heavily crowds out private employment
\end{itemize}
\end{itemize}
\end{frame}

\begin{frame}
\frametitle{what happens if $\a=1$?}
\begin{itemize}
\item $\a=1$: linear production function
\begin{itemize}
\item standard assumption [Pissarides 2000; Hall 2005]
\end{itemize}
\item in $(n,\t)$ plane: labor demand is horizontal
\item[\then] a change in $g$ does not change $\t$
\item[\then] crowding out is one-for-one
\item[\then]  multiplier $=0$
\end{itemize}
\end{frame}

\begin{frame}
\frametitle{what happens if $\g=1$?}
\begin{itemize}
\item $\g=1$: flexible wage
\begin{itemize}
\item as with Nash bargaining
\end{itemize}
\item in $(n,\t)$ plane: labor demand is independent of $a$
\item[\then] $\t$ is independent of $a$
\item[\then] crowding out is independent of $a$
\item[\then] multiplier is acyclical
\end{itemize}
\end{frame}

\begin{frame}
\heading{New Keynesian model}
\end{frame}


\begin{frame}
\frametitle{standard features}
\begin{itemize}
\item fluctuations arise from technology shocks
\item representative large household 
\begin{itemize}
\item works for intermediate-good firms
\item consumes final good
\item saves using nominal bonds
\end{itemize}
\item representative final-good firm
\begin{itemize}
\item uses intermediate goods as input
\item sells output on perfectly competitive market
\end{itemize}
\end{itemize}
\end{frame}

\begin{frame}
\frametitle{standard features}
\begin{itemize}
\item intermediate-good firms
\begin{itemize}
\item use labor as input
\item sell output on monopolistically competitive market to final-good firm
\item set price subject to a price-setting friction
\end{itemize} 
\item monetary policy
\begin{itemize}
\item interest-rate rule (Taylor rule)
\end{itemize}
\end{itemize}
\end{frame}

\begin{frame}
\frametitle{nonstandard features}
\begin{itemize}
\item labor market with matching structure from \href{https://www.pascalmichaillat.org/1.html}{Michaillat [2012]}
\begin{itemize}
\item instead of perfect/monopolistic competition
\end{itemize}
\item quadratic price-adjustment cost from Rotemberg [1982] 
\begin{itemize}
\item instead of Calvo [1983] pricing
\end{itemize}
\item government consumption is public employment
\begin{itemize}
\item instead of purchase of goods
\end{itemize}
\end{itemize}
\end{frame}


\begin{frame}
\frametitle{9 endogenous variables}
\begin{itemize}
\item exogenous variables:
\begin{equation*}
\bc{a_{t},g_{t}}_{t=0}^{+\infty}
\end{equation*}
\item endogenous variables:
\begin{equation*}
\bc{\t_{t},n_{t},l_{t},w_{t},\L_{t},c_{t},y_{t},R_{t},\pi_{t}}_{t=0}^{+\infty}
\end{equation*}
\end{itemize}
\end{frame}

\begin{frame}
\frametitle{labor market equations}
\begin{itemize}
\item equation \#1: wage schedule
\begin{equation*}
w_{t}=\o\cdot a_{t}^{\g},\;\g<1
\end{equation*}
\item equation \#2: labor supply
\begin{equation*}
n_{t}=(1-s)\cdot n_{t-1}+f(\t_{t})\cdot \bs{1-(1-s)\cdot n_{t-1}}
\end{equation*}
\item equation \#3: public-employment policy
\begin{equation*}
n_{t}=l_{t}+g_{t}
\end{equation*}
\end{itemize}
\end{frame}


\begin{frame}
\frametitle{production equations}
\begin{itemize}
\item equation \#4: production function
\begin{equation*}
y_{t}=a_{t}\cdot l_{t}^{\a},\;\a<1
\end{equation*}
\item equation \#5: resource constraint
\begin{equation*}
y_{t}-\frac{r\cdot a_{t}}{q(\t_{t})}\cdot\bs{n_{t}-(1-s)\cdot n_{t-1}}=c_{t}\cdot \bs{1+\frac{\f}{2}\cdot \pi_{t}^{2}}
\end{equation*}
\end{itemize}
\end{frame}

\begin{frame}
\frametitle{bond market equations}
\begin{itemize}
\item equation \#6: Euler equation
\begin{equation*}
1=\b\cdot \E[t]{\frac{R_{t}}{1+\pi_{t+1}}\cdot\frac{c_{t}}{c_{t+1}}}
\end{equation*}
\item equation \#7: Taylor rule
\begin{equation*}
R_{t}=\frac{1}{\b}\cdot \bp{1+\pi_{t}}^{\mu_{\pi}\cdot (1-\mu_{R})}\cdot \bp{\b\cdot R_{t-1}}^{\mu_{R}}
\end{equation*}
\end{itemize}
\end{frame}


\begin{frame}
\frametitle{firm equations}
\begin{itemize}
\item equation \#8: optimal pricing decision
\begin{equation*}
\pi_{t}\cdot\bp{\pi_{t}+1}=\frac{1}{\f} \cdot\frac{y_{t}}{c_{t}} \bs{\e \cdot\L_{t}-\bp{\e-1}}+\b\cdot\E[t]{\pi_{t+1}\cdot \bp{\pi_{t+1}+1}}
\end{equation*}
\item equation \#9: optimal employment decision
\begin{equation*}
\L_{t}\cdot \a\cdot l_{t}^{\a-1} = \frac{w_{t}}{a_{t}}+\frac{r}{q(\t_{t})}-\b \cdot (1-s)\cdot  \E[t]{\frac{c_{t}}{c_{t+1}}\cdot \frac{a_{t+1}}{a_{t}}\cdot \frac{r}{q(\t_{t+1})}}
\end{equation*}
\end{itemize}
\end{frame}

\begin{frame}
\frametitle{steady state $(n,\t)$ with zero inflation}
\begin{itemize}
\item equation \#2: labor supply
\begin{equation*}
n^{s}(\t)=\frac{f(\t)}{s+(1-s)\cdot f(\t)}
\end{equation*}
\item equation \#8: $\L=(\e-1)/\e$
\item equation \#1: $w=\o\cdot a^{\g}$
\item equation \#9: firms' labor demand
\begin{equation*}
\frac{\e-1}{\e}\cdot \a\cdot \bs{l^{d}(\t,a)}^{\a-1}=\o\cdot a^{\g-1}+\bp{1-\b \cdot (1-s)}\cdot\frac{r}{q(\t)}
\end{equation*}
\end{itemize}
\end{frame}

\begin{frame}
\heading{simulations}
\end{frame}

\begin{frame}
\frametitle{simulation method}
\begin{itemize}
\item simulate nonlinear model under perfect foresight using shooting algorithm
\item scenario \#1: \bal{public employment without stimulus}
\begin{itemize}
\item value of $g$: $\bal{\hat{g}_{t} = \ol{g}}$
\item value of any $x$: $\bal{\hat{x}_{t}}$
\item \bal{solid blue lines in graphs}
\end{itemize}
\item scenario \#2: \ral{public employment with stimulus}
\begin{itemize}
\item value of $g$: $\ral{g^{*}_{t}>\ol{g}}$
\item value of any $x$: $\ral{x^{*}_{t}}$
\item \ral{dashed red lines in graphs}
\end{itemize}
\end{itemize}
\end{frame}


\begin{frame}
\frametitle{computation of the multiplier}
\begin{itemize}
\item instantaneous multiplier in a simulation:
\begin{equation*}
\frac{\ral{n^{*}_{t}}-\bal{\hat{n}_{t}}}{\ral{g^{*}_{t}}-\bal{\hat{g}_{t}}}
\end{equation*}
\item cumulative multiplier in a simulation:
\begin{equation*}
\frac{\sum_{t=0}^{T}\ral{n^{*}_{t}}-\bal{\hat{n}_{t}}}{\sum_{t=0}^{T} \ral{g^{*}_{t}}-\bal{\hat{g}_{t}}}
\end{equation*}
\item cumulative multipliers are parametrized by the peak of the unemployment rate in the
simulation
\end{itemize}
\end{frame}

\begin{frame}
\frametitle{response to positive technology shock}
\includegraphics[scale=\sfig,page=13]{\pdf}%
\end{frame}

\begin{frame}
\frametitle{multiplier after positive shock}
\includegraphics[scale=\sfig,page=14]{\pdf}%
\end{frame}

\begin{frame}
\frametitle{response to negative technology shock}
\includegraphics[scale=\sfig,page=15]{\pdf}%
\end{frame}

\begin{frame}
\frametitle{multiplier after negative shock}
\includegraphics[scale=\sfig,page=16]{\pdf}%
\end{frame}

\begin{frame}
\frametitle{countercyclical cumulative multiplier}
\includegraphics<1>[scale=\sfig,page=17]{\pdf}%
\includegraphics<2>[scale=\sfig,page=18]{\pdf}%
\end{frame}

\begin{frame}
\heading{conclusion}
\end{frame}

\begin{frame}
\frametitle{summary}
\begin{itemize}
\item this paper proposes a New Keynesian model in which the government multiplier doubles when unemployment rises from 5\% to 8\% 
\item mechanism behind countercyclical multiplier: 
\begin{itemize}
\item  multiplier $= 1 -$ crowding out
\item crowding out of private employment by public employment is much weaker when unemployment is higher
\end{itemize}
\end{itemize}
\end{frame}

\begin{frame}
\frametitle{applications}
\begin{itemize}
\item the same mechanism explains the procyclicality of the macroelasticity of unemployment with respect to unemployment insurance
\begin{itemize}
\item see \href{https://www.pascalmichaillat.org/5.html}{Landais, Michaillat, \& Saez [2018]}
\end{itemize}
\item the same mechanism applies to the product market
\begin{itemize}
\item see \href{https://www.pascalmichaillat.org/6.html}{Michaillat \& Saez [2019]}
\end{itemize} 
\item the multiplier determines optimal stimulus spending
\begin{itemize}
\item  see \href{https://www.pascalmichaillat.org/6.html}{Michaillat \& Saez [2019]}
\end{itemize}
\end{itemize}
\end{frame}

\end{document}
