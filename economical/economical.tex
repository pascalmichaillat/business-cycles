\documentclass[12pt,xcolor={dvipsnames},hyperref={pdftex,pdfpagemode=UseNone,hidelinks,pdfdisplaydoctitle=true},usepdftitle=false]{beamer}
\usepackage{presentation,math}
\def\pdf{xeconomical.pdf}
\hypersetup{pdftitle={Lecture | An Economical Business-Cycle Model}}

\begin{document}

\title{An Economical Business-Cycle Model}
\information{Pascal Michaillat, Emmanuel Saez}%
{Oxford Economic Papers, 2021}%
{https://www.pascalmichaillat.org/7.html}
\frame[plain]{\titlepage}

\begin{frame}
\frametitle{Limitations of the New Keynesian model}
\begin{enumerate}
\item lacks conceptual economy
\begin{itemize}
\item not taught to undergraduates
\item not used in related fields, outside of macroeconomics
\item not used by policymakers for day-to-day thinking (Krugman 2000, 2018)
\end{itemize}
\item does not not describe business cycles well
\begin{itemize}
\item does not feature unemployment
\item makes anomalous predictions about long-lasting ZLB episodes \href{https://www.pascalmichaillat.org/11.html}{(Michaillat, Saez 2021)}
\end{itemize}
\end{enumerate}
\end{frame}

\begin{frame}
\frametitle{This paper's business-cycle model}
\begin{enumerate}
\item is more economical
\begin{itemize}
\item solved with an AD-AS diagram
\item effects of shocks derived by comparative statics
\item efficient unemployment \& optimal policies described by sufficient-statistic formulas
\item most complicated step: derivation of Euler equation
\end{itemize}
\item describes business cycles better
\begin{itemize}
\item features unemployment: fluctuating \& generally inefficient
\item behaves well during long/permanent ZLB episodes
\end{itemize}
\end{enumerate}
\end{frame}

\begin{frame}
\heading{Assumptions}
\end{frame}

\begin{frame}
\frametitle{Service economy, without firms}
\includegraphics<1>[scale=\sfig,page=1]{\pdf}%
\includegraphics<2>[scale=\sfig,page=2]{\pdf}%
\end{frame}

\begin{frame}
\frametitle{Matching function \href{https://www.pascalmichaillat.org/3.html}{(Michaillat, Saez 2015)}}
\includegraphics<1>[scale=\sfig,page=3]{\pdf}%
\includegraphics<2>[scale=\sfig,page=43]{\pdf}%
\includegraphics<3>[scale=\sfig,page=44]{\pdf}%
\end{frame}

\begin{frame}
\frametitle{Wealth in utility \href{https://www.pascalmichaillat.org/11.html}{(Michaillat, Saez 2021)}}
\includegraphics<1>[scale=\sfig,page=4]{\pdf}%
\includegraphics<2>[scale=\sfig,page=5]{\pdf}%
\includegraphics<3>[scale=\sfig,page=6]{\pdf}%
\end{frame}

\begin{frame}
\heading{Solution}
\end{frame}

\begin{frame}
\frametitle{Matching function \then Beveridge curve}
\includegraphics<1>[scale=\sfig,page=7]{\pdf}%
\end{frame}

\begin{frame}
\frametitle{Unemployment: always on Beveridge curve}
\includegraphics<1>[scale=\sfig,page=8]{\pdf}%
\end{frame}

\begin{frame}
\frametitle{Beveridge curve \then Aggregate supply}
\includegraphics<1>[scale=\sfig,page=9]{\pdf}%
\includegraphics<2>[scale=\sfig,page=10]{\pdf}%
\end{frame}

\begin{frame}
\frametitle{Wealth in utility \then Euler equation}
\includegraphics<1>[scale=\sfig,page=11]{\pdf}%	
\end{frame}

\begin{frame}
\frametitle{Euler equation \then Aggregate demand}
\includegraphics<1>[scale=\sfig,page=12]{\pdf}%
\end{frame}

\begin{frame}
\frametitle{Price norm: fixed inflation}
\begin{itemize}
\item any model with a matching function needs a price mechanism
\item we assume that prices grow at a fixed rate of inflation
\begin{itemize}
\item interpretation: fixed inflation is a social norm (Hall 2005)
\end{itemize}
\item fixed inflation is realistic:
\begin{itemize}
\item inflation does not respond to unemployment (Stock, Watson 2010, 2019)
\item inflation does not respond to monetary policy (Christiano, Eichenbaum, Evans 1999)
\end{itemize}
\item fixed inflation does not create bilaterally inefficiencies: 
\begin{itemize}
\item buyers \& sellers are happy to transact at the given price
\end{itemize}
\end{itemize}    
\end{frame}

\begin{frame}
\frametitle{Solution of the model}
\includegraphics<1>[scale=\sfig,page=13]{\pdf}%
\includegraphics<2>[scale=\sfig,page=14]{\pdf}%
\end{frame}

\begin{frame}
\frametitle{Keynesian vs. frictional unemployment}
\includegraphics<1>[scale=\sfig,page=15]{\pdf}%
\includegraphics<2>[scale=\sfig,page=16]{\pdf}%
\end{frame}

\begin{frame}
\heading{Inefficiency}
\end{frame}

\begin{frame}
\frametitle{Efficient allocation \href{https://www.pascalmichaillat.org/9.html}{(Michaillat, Saez 2020)}}
\includegraphics<1>[scale=\sfig,page=17]{\pdf}%
\includegraphics<2>[scale=\sfig,page=18]{\pdf}%
\includegraphics<3>[scale=\sfig,page=19]{\pdf}%
\includegraphics<4>[scale=\sfig,page=20]{\pdf}%
\end{frame}

\begin{frame}
\frametitle{Inefficient allocations}
\includegraphics<1>[scale=\sfig,page=21]{\pdf}%
\includegraphics<2>[scale=\sfig,page=22]{\pdf}%
\end{frame}

\begin{frame}
\frametitle{Efficient tightness}
\includegraphics<1>[scale=\sfig,page=23]{\pdf}%
\end{frame}

\begin{frame}
\heading{Business cycles}
\end{frame}

\begin{frame}
\frametitle{Negative demand shock}
\includegraphics<1>[scale=\sfig,page=24]{\pdf}%
\includegraphics<2>[scale=\sfig,page=25]{\pdf}%
\end{frame}

\begin{frame}
\frametitle{Negative supply shock}
\includegraphics<1>[scale=\sfig,page=26]{\pdf}%
\includegraphics<2>[scale=\sfig,page=27]{\pdf}%
\end{frame}

\begin{frame}
\frametitle{Okun's law \so demand shocks are prevalent}
\includegraphics<1>[scale=0.3,page=45]{\pdf}%
\includegraphics<2>[scale=0.3,page=46]{\pdf}%
\end{frame}

\begin{frame}
\heading{Monetary policy}
\end{frame}

\begin{frame}
\frametitle{Reduction in interest rate}
\includegraphics<1>[scale=\sfig,page=28]{\pdf}%
\includegraphics<2>[scale=\sfig,page=29]{\pdf}%
\end{frame}

\begin{frame}
\frametitle{Zero lower bound}
\includegraphics<1>[scale=\sfig,page=30]{\pdf}%
\includegraphics<2>[scale=\sfig,page=31]{\pdf}%
\end{frame}

\begin{frame}
\frametitle{Increase in wealth tax}
\includegraphics<1>[scale=\sfig,page=32]{\pdf}%
\includegraphics<2>[scale=\sfig,page=33]{\pdf}%
\end{frame}

\begin{frame}
\frametitle{Optimal monetary policy: boom}
\includegraphics<1>[scale=\sfig,page=34]{\pdf}%
\includegraphics<2>[scale=\sfig,page=35]{\pdf}%
\end{frame}

\begin{frame}
\frametitle{Optimal monetary policy: small slump}
\includegraphics<1>[scale=\sfig,page=36]{\pdf}%
\includegraphics<2>[scale=\sfig,page=37]{\pdf}%
\includegraphics<3>[scale=\sfig,page=38]{\pdf}%
\end{frame}

\begin{frame}
\frametitle{Optimal monetary policy: large slump}
\includegraphics<1>[scale=\sfig,page=39]{\pdf}%
\includegraphics<2>[scale=\sfig,page=40]{\pdf}%
\end{frame}

\begin{frame}
\frametitle{Large slump: role for wealth tax}
\includegraphics<1>[scale=\sfig,page=41]{\pdf}%
\end{frame}

\begin{frame}
\frametitle{Monetary multiplier:  $du/di = 0.5$}
\begin{table}
\begin{tabular*}{\textwidth}{@{\extracolsep\fill}lcc}
study &   $du/di$ & method \\
\toprule
Bernanke, Blinder (1992) & $0.6$ & VAR  \\
Leeper, Sims, Zha (1996) & $0.1$ & VAR \\
Christiano, Eichenbaum, Evans (1996) &  $0.1$ & VAR \\
Romer, Romer (2003)  &  $0.9$ & narrative \\
Bernanke, Boivin, Eliasz (2005) &  $0.2$ & FAVAR\\
Coibion (2012)  &  $0.5$ & narrative \& VAR \\
\bottomrule
\end{tabular*}
\end{table}
\end{frame}

\begin{frame}
\frametitle{Unemployment gap: \href{https://www.pascalmichaillat.org/9.html}{Michaillat, Saez (2020)}}
\includegraphics<1>[scale=\sfig,page=42]{\pdf}%
\end{frame}

\begin{frame}
\frametitle{Optimal monetary policy formula}
\begin{itemize}
\item linear expansion around suboptimal $\bs{i,u}$ assessed at optimal $\bs{i^*,u^*}$:
$u^* \approx u + (\odx{u}{i}) \cdot (i^* - i)$
\item sufficient-statistic formula:
\begin{equation*}
i - i^* \approx  \frac{u-u^*}{du/di}
\end{equation*}
\item[\then] Fed should reduce interest rate by 2 percentage points for each percentage point of unemployment gap
\item[\then] in line with observed Fed behavior (Bernanke, Blinder 1992)
\end{itemize}
\end{frame}

\begin{frame}
\heading{Conclusion}
\end{frame}

\begin{frame}
\frametitle{Summary of model properties}
\begin{table}
\small
\begin{tabular*}{\textwidth}{@{\extracolsep\fill}lcc}
property &  NK model & this model \\
\toprule
AD relation & Euler equation & discounted Euler equation \\
AS relation & Phillips curve  & Beveridge curve \\
inflation &  fluctuating & fixed \\
unemployment &  zero & fluctuating \\
ZLB world & topsy-turvy & normal \\
ZLB duration  & must be short & can be permanent \\
\bottomrule
\end{tabular*}
\end{table}
\end{frame}

\begin{frame}
\frametitle{Summary of monetary policy properties}
\begin{table}
\small
\begin{tabular*}{\textwidth}{@{\extracolsep\fill}lcc}
property &  NK model & this model \\
\toprule
response to inflation & must be strong & not required \\
 & (Taylor principle) & (interest-rate peg works) \\
policy target & inflation rate  & unemployment rate \\
optimal rule &  not implementable & implementable  \\
& & w/ sufficient statistics\\
multiplier $du/di$ & useless & key statistic \\
forward guidance & very powerful & less \& less potent \\
 & at ZLB & as ZLB lasts longer\\
isomorphic policy  & -- & wealth tax \\
\bottomrule
\end{tabular*}
\end{table}
\end{frame}

\begin{frame}
\frametitle{Other policies}
\begin{itemize}
\item public hiring or spending (\href{https://www.pascalmichaillat.org/2.html}{Michaillat 2014}; \href{https://www.pascalmichaillat.org/6.html}{Michaillat, Saez 2019})
\begin{itemize}
\item multiplier is higher when unemployment is higher 
\item optimal policy deviates from the Samuelson rule to reduce the unemployment gap 
\end{itemize}
\item unemployment insurance \href{https://www.pascalmichaillat.org/4.html}{ (Landais, Michaillat, Saez 2018)}
\begin{itemize}
\item optimal policy deviates from the Baily-Chetty rule to reduce the tightness gap 
\end{itemize}
\end{itemize}	
\end{frame}

\end{document}