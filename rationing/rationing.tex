\documentclass[12pt,xcolor={dvipsnames},hyperref={pdftex,pdfpagemode=UseNone,hidelinks,pdfdisplaydoctitle=true},usepdftitle=false]{beamer}
\usepackage{presentation,math}
\def\pdf{xrationing.pdf}
\hypersetup{pdftitle={Lecture | Do Matching Frictions Explain Unemployment? Not in Bad Times}}

\begin{document}

\title{Do Matching Frictions Explain Unemployment? Not in Bad Times}
\information{Pascal Michaillat}%
{American Economic Review, 2012}%
{https://www.pascalmichaillat.org/1.html}

\frame[plain]{\titlepage}

\begin{frame}
\frametitle{workers queue for jobs in bad times}
\begin{figure}
\includegraphics<1>[scale=\sfig,page=31]{\pdf}%
\includegraphics<2>[scale=\sfig,page=32]{\pdf}%
\includegraphics<3>[scale=\sfig,page=33]{\pdf}%
\end{figure}
\end{frame}

\begin{frame}
\frametitle{existing matching models: no queues}
\begin{itemize}
\item a queue is a situation where workers desperately want a job but cannot find one  
\item in existing models, unemployment vanishes when workers desperately want a job \then queues cannot exist
\begin{itemize}
\item formally: unemployment vanishes when workers' job-search effort becomes infinite
\end{itemize}
\item problem with existing models: firms hire everybody when recruiting is costless
\end{itemize}
\end{frame}

\begin{frame}
\frametitle{this paper: matching model with queues}
\begin{itemize}
\item firms may not hire everybody when recruiting is costless
\item based on two assumptions:
\begin{itemize}
\item diminishing marginal returns to labor
\item wage rigidity
\end{itemize}
\item in bad times, jobs are rationed:
\begin{itemize}
\item unemployment would not disappear if recruiting costs vanished
\item queues could appear
\end{itemize}
\end{itemize}
\end{frame}

\begin{frame}
\heading{generic matching model}
\end{frame}

\begin{frame}
\frametitle{matching function}
\begin{figure}
\includegraphics<1>[scale=\sfig,page=1]{\pdf}%
\includegraphics<2>[scale=\sfig,page=2]{\pdf}%
\includegraphics<3>[scale=\sfig,page=3]{\pdf}%
\end{figure}
\end{frame}

\begin{frame}
\frametitle{worker flows: job creation \& destruction}
\begin{figure}
\includegraphics<1>[scale=\sfig,page=4]{\pdf}%
\includegraphics<2>[scale=\sfig,page=5]{\pdf}%
\includegraphics<3>[scale=\sfig,page=6]{\pdf}%
\end{figure}
\end{frame}

\begin{frame}
\frametitle{Beveridge curve}
\begin{itemize}
\item the Beveridge curve relates employment $n$ to tightness $\t$ when labor market flows are balanced
\begin{itemize}
\item E $\to$ U = U $\to$ E
\item $s\cdot n = f(\t)\cdot u = f(\t)\cdot \bs{1-n+s\cdot n}$
\end{itemize}
\item equation of the Beveridge curve:
\begin{equation*}
n=\frac{f(\t)}{s+(1-s)\cdot f(\t)}
\end{equation*}
\end{itemize}
\end{frame}

\begin{frame}
\frametitle{generic wage schedule}
\begin{itemize}
\item there are mutual gains from matching
\item many wage schedules are consistent with equilibrium
\item generic wage schedule: $w_t= w(n_t,\t_t,x_t)$
\begin{itemize}
\item $n_{t}$: level of employment in the firm
\item $\t_{t}$: aggregate level of tightness
\item $x_{t}$: state of the economy
\end{itemize}
\item $w$ nests various types of bargaining and wage rigidity
\end{itemize}
\end{frame}

\begin{frame}
\frametitle{representative firm}
\begin{itemize}
\item employs $n_{t}$ workers paid $w_{t}$
\item produces $y_{t}= g(n_{t},a_{t})$
\begin{itemize}
\item $g$: production function
\item $a_{t}$: productivity (random variable)
\end{itemize}
\item hires $n_{t}-(1-s)\cdot n_{t-1}$ new workers
\begin{itemize}
\item cost per vacancy: $c\cdot  a_{t}$
\item probability to fill a vacancy: $q(\t_{t})$
\end{itemize}
\end{itemize}
\end{frame}

\begin{frame}
\frametitle{firm problem}
\begin{itemize}
\item given productivity $\bc{a_t}$, tightness $\bc{\t_t}$, and the wage schedule $w$, the firm chooses employment $\bc{n_{t}}$ to maximize expected profits
\begin{align*}\hspace*{-1cm}
\E_0 \sum_{t=0}^{+\infty} \d^t\left[\underbrace{g(n_{t},a_{t})}_{\text{production}}\right.&-\underbrace{w(n_t,\t_t,x_t)\cdot  n_t}_{\text{wage bill}}\\
-&\left.\underbrace{\frac{c \cdot a_{t}}{q(\t_{t})}\cdot \bp{n_{t}-(1-s)\cdot  n_{t-1}}}_{\text{recruiting expenses}}\right]
\end{align*}
\end{itemize}
\end{frame}


\begin{frame}
\frametitle{profit maximization}
\begin{equation*}
\pd{g(n,a)}{n}- w- n \cdot \pd{w(n,\t,x)}{n}-\bs{1-\d \cdot (1-s)}\cdot \frac{c \cdot a}{q(\t)}=0
\end{equation*}
\begin{itemize}
\item the condition says that marginal profit = 0
\item the marginal profit is the sum of 
\begin{itemize}
\item gross marginal profit: independent of $c$
\item marginal recruiting expenses: dependent on $c$
\end{itemize}
\item (this is the steady-state expression of the condition)
\end{itemize}
\end{frame}

\begin{frame}
\heading{absence or presence of job rationing in several models}
\end{frame}

\begin{frame}
\frametitle{definition of job rationing}
\begin{itemize}
\item jobs are rationed if the employment rate remains strictly below 1 when recruiting is costless
\item equivalently, jobs are rationed if the employment rate remains strictly below 1 when the recruiting cost $c\to 0$ 
\item when jobs are rationed, queues could exist
\begin{itemize}
\item employment is the same when job-search effort $\to \infty$ and when $c\to 0$
\end{itemize}
\end{itemize}
\end{frame}

\begin{frame}
\frametitle{four matching models}
\begin{table}
\begin{tabular*}{\textwidth}{@{\extracolsep\fill}lcc}
 model &  production function & wage setting \\
\toprule
Pissarides [2000]  & constant returns & Nash bargaining \\
 & to labor &  \\
\midrule
Cahuc \& & diminishing marginal & Stole-Zwiebel bargaining  \\
Wasmer [2001] & returns to labor &  \\
\midrule
Hall [2005]  &  constant returns & rigid wage  \\
 &  to labor &  \\
\midrule
this paper & diminishing marginal & rigid wage  \\
&  returns to labor & \\
\bottomrule
\end{tabular*}
\end{table}
\end{frame}

\begin{frame}
\frametitle{the model of Pissarides [2000]}
\begin{itemize}
\item linear production function: $g(n,a)=a\cdot n$
\item wage from Nash bargaining:
\begin{equation*}
w=a\cdot c\cdot \frac{ \b}{1-\b} \bs{\frac{1-\d\cdot (1-s)}{q(\t)}+ \d\cdot 
(1-s)\cdot \t}
\end{equation*}
\begin{itemize}
\item $\b\in(0,1)$: workers' bargaining power
\item (this is the steady-state expression of the wage)
\end{itemize}
\end{itemize}
\end{frame}

\begin{frame}
\frametitle{Pissarides [2000]: equilibrium}
\only<1>{\begin{itemize}
\item steady-state equilibrium: pair $(n,\t)$ that satisfies
\begin{itemize}
\item Beveridge curve
\item firm's profit-maximization condition
\end{itemize} 
\item equilibrium condition:
\begin{equation*}\hspace*{-1cm}
\underbrace{1-\beta}_{\text{gross marginal profit}}=\underbrace{c\cdot  \bs{\frac{1-\d\cdot (1-s)}{q(\t(n))}+\d\cdot (1-s)\cdot \b\cdot \t(n)}}_{\text{marginal recruiting expenses}}
\end{equation*}
\begin{itemize}
\item where $\t(n)$ is implicitly defined by Beveridge curve
\end{itemize}
\end{itemize}}
\only<2>{\begin{figure}
\includegraphics[scale=\sfig,page=7]{\pdf}%
\end{figure}}
\end{frame}

\begin{frame}
\frametitle{Pissarides [2000]: equilibrium as $c\to 0$}
\begin{figure}
\includegraphics[scale=\sfig,page=8]{\pdf}%
\end{figure}
\end{frame}

\begin{frame}
\frametitle{Pissarides [2000]: no job rationing}
\begin{figure}
\includegraphics[scale=\sfig,page=9]{\pdf}%
\end{figure}
\end{frame}


\begin{frame}
\frametitle{the model of Cahuc \& Wasmer [2001]}
\begin{itemize}
\item concave production function: $g(n,a)=a\cdot n^{\a}$
\begin{itemize}
\item $\a<1$: diminishing marginal returns to labor
\end{itemize}
\item wage from Stole-Zwiebel bargaining:
\begin{equation*}
w=a\cdot \bs{ \frac{ \b\cdot\a}{1-\b\cdot (1-\a)}\cdot n^{\a-1}+c\cdot (1-s)\cdot \d \cdot \b
\cdot \t}
\end{equation*}
\begin{itemize}
\item $\b\in(0,1)$: workers' bargaining power
\item (this is the steady-state expression of the wage)
\end{itemize}
\end{itemize}
\end{frame}

\begin{frame}
\frametitle{Cahuc \& Wasmer [2001]: equilibrium}
\only<1>{\begin{itemize}
\item steady-state equilibrium: pair $(n,\t)$ that satisfies
\begin{itemize}
\item Beveridge curve
\item firm's profit-maximization condition
\end{itemize} 
\item equilibrium condition:
\begin{equation*}
\underbrace{\frac{\a\cdot (1-\b)}{1-\b\cdot (1-\a)}\cdot  n^{\a-1}}_{\text{gross marginal profit}}=\underbrace{c\cdot \bs{\frac{1-\d (1-s)}{q(\t(n))}+\d (1-s)\cdot \b\cdot \t(n)}}_{\text{marginal recruiting expenses}}
\end{equation*}
\begin{itemize}
\item where $\t(n)$ is implicitly defined by Beveridge curve
\end{itemize}
\end{itemize}}
\only<2>{\begin{figure}
\includegraphics[scale=\sfig,page=10]{\pdf}%
\end{figure}}
\end{frame}

\begin{frame}
\frametitle{Cahuc \& Wasmer [2001]: no job rationing}
\begin{figure}
\includegraphics[scale=\sfig,page=11]{\pdf}%
\end{figure}
\end{frame}

\begin{frame}
\frametitle{the model of Hall [2005]}
\begin{itemize}
\item linear production function: $g(n,a)=a\cdot n$
\item rigid wage: $w= \o\cdot a^{\g}$
\begin{itemize}
\item $\o>0$: level of the real wage
\item $\g<1$: partially rigid real wage
\item if $\g=0$: fixed wage
\item specification from Blanchard \& Gali [2010]
\end{itemize}
\end{itemize}
\end{frame}

\begin{frame}
 \frametitle{Hall [2005]: equilibrium}
\only<1>{\begin{itemize}
\item steady-state equilibrium: pair $(n,\t)$ that satisfies
\begin{itemize}
\item Beveridge curve
\item firm's profit-maximization condition
\end{itemize} 
\item equilibrium condition:
\begin{equation*}
\underbrace{1-\o\cdot a^{\g-1}}_{\text{gross marginal profit}}=\underbrace{c\cdot  \frac{1-\d\cdot (1-s)}{q(\t(n))}}_{\text{marginal recruiting expenses}}
\end{equation*}
\begin{itemize}
\item where $\t(n)$ is implicitly defined by Beveridge curve
\end{itemize}
\end{itemize}}
\only<2>{\begin{figure}
\includegraphics[scale=\sfig,page=12]{\pdf}%
\end{figure}}
\end{frame}

\begin{frame}
\frametitle{Hall [2005]: no job rationing}
\begin{figure}
\includegraphics[scale=\sfig,page=13]{\pdf}%
\end{figure}
\end{frame}

\begin{frame}
 \frametitle{this paper's model}
\begin{itemize}
\item concave production function: $g(n,a)=a\cdot n^{\a}$
\begin{itemize}
\item $\a<1$: diminishing marginal returns to labor
\end{itemize}
\item rigid wage: $w= \o\cdot a^{\g}$
\begin{itemize}
\item $\o>0$: level of the real wage
\item $\g<1$: partially rigid real wage
\item if $\g=0$: fixed wage
\item specification from Blanchard \& Gali [2010]
\end{itemize}
\end{itemize}
\end{frame}

\begin{frame}
\frametitle{this paper's model: equilibrium}
\only<1>{\begin{itemize}
\item steady-state equilibrium: pair $(n,\t)$ that satisfies
\begin{itemize}
\item Beveridge curve
\item firm's profit-maximization condition
\end{itemize} 
\item equilibrium condition:
\begin{equation*}
\underbrace{\a\cdot n^{\a-1}-\o\cdot a^{\g-1}}_{\text{gross marginal profit}}=\underbrace{c\cdot  \frac{1-\d\cdot (1-s)}{q(\t(n))}}_{\text{marginal recruiting expenses}}
\end{equation*}
\begin{itemize}
\item where $\t(n)$ is implicitly defined by Beveridge curve
\end{itemize}
\end{itemize}}
\only<2>{\begin{figure}
\includegraphics[scale=\sfig,page=14]{\pdf}%
\end{figure}}
\end{frame}

\begin{frame}
\frametitle{this paper's model: equilibrium as $c\to 0$}
\begin{figure}
\includegraphics[scale=\sfig,page=15]{\pdf}%
\end{figure}
\end{frame}

\begin{frame}
\frametitle{this paper's model: job rationing}
\begin{figure}
\includegraphics<1>[scale=\sfig,page=16]{\pdf}%
\includegraphics<2>[scale=\sfig,page=17]{\pdf}%
\end{figure}
\end{frame}

\begin{frame}
\frametitle{frictional \& rationing unemployment}
\begin{figure}
\includegraphics<1>[scale=\sfig,page=18]{\pdf}%
\includegraphics<2>[scale=\sfig,page=19]{\pdf}%
\end{figure}
\end{frame}

\begin{frame}
\frametitle{summary}
\renewcommand{\arraystretch}{0.75}
\begin{table}
\begin{tabular*}{\textwidth}{@{\extracolsep\fill}lcc}
model  & assumptions & job rationing?\\
\toprule
Pissarides [2000] & bargaining  &  \ral{no}\\
 & linear production & \\
\midrule
Cahuc \&  & bargaining & \ral{no} \\
Wasmer [2001] & concave production &\\
\midrule
Hall [2005] & rigid wage & \ral{no}\\
& linear production & \\
\midrule
this paper  & rigid wage &\gal{yes} \\
& concave production &  \\
\bottomrule
\end{tabular*}
\end{table}
\end{frame}


\begin{frame}
\heading{frictional unemployment over the business cycle: comparative statics}
\end{frame}

\begin{frame}
\frametitle{frictional unemployment is high in booms}
\begin{figure}
\includegraphics[scale=\sfig,page=20]{\pdf}%
\end{figure}
\end{frame}

\begin{frame}
\frametitle{frictional unemployment is low in slumps}
\begin{figure}
\includegraphics[scale=\sfig,page=21]{\pdf}%
\end{figure}
\end{frame}

\begin{frame}
\frametitle{summary}
\begin{itemize}
\item with low productivity, gross marginal profits are low
\begin{itemize}
\item because of wage rigidity
\end{itemize}
\item[\then] labor demand is depressed
\item[\then] total unemployment \& rationing unemployment are high
\item but it is easy for firms to recruit workers 
\item[\then] frictional unemployment is low
\end{itemize}
\end{frame}

\begin{frame}
\heading{frictional unemployment over the business cycle: simulations}
\end{frame}

\begin{frame}
\frametitle{calibration (weekly frequency)}
\begin{table}
\begin{footnotesize}
\begin{tabular*}{\textwidth}{@{\extracolsep\fill}lccc}
& interpretation & value &  source\\
\toprule
$\eta$ & elasticity of matching  & $0.5$ & Petrongolo \& Pissarides [2001]\\
\al{$\g$} & \al{real wage flexibility} & \al{$0.7$} & \al{Haefke et al [2008]}\\
$c$ & recruiting cost  & $0.22$   & Barron et al [1997]\\
&&& Silva \& Toledo [2009] \\
$s$ & separation rate   & $0.95\%$     & JOLTS, 2000--2009 \\
$\mu$ & effectiveness of matching &  $0.23$  & JOLTS, 2000--2009 \\
$\a$ & marginal returns to labor & $0.67$ & matches labor share $=0.66$ \\
$\o$ & steady-state real wage &$0.67$& matches unemployment $=5.8\%$\\
$\rho$ & autocorrelation of productivity &$0.992$& MSPC, 1964--2009\\
$\o$ & standard deviation of shocks &$0.0027$&  MSPC, 1964--2009\\
\bottomrule
\end{tabular*}
\end{footnotesize}
\end{table}
\end{frame}

\begin{frame}
\frametitle{impulse responses to negative shock}
\begin{figure}
\includegraphics[scale=\sfig,page=22]{\pdf}%
\end{figure}
\end{frame}

\begin{frame}
\frametitle{simulated \& empirical moments}
\only<1>{\begin{table}
\begin{tabular*}{\textwidth}{@{\extracolsep\fill}lcc}
moment & model & US data \\
\toprule
elasticity of $u$ wrt $a$ & \al{$5.9$} & \al{$4.2$}\\
elasticity of $v$ wrt $a$ & \al{$6.8$} & \al{$4.3$}\\
elasticity of $w$ wrt $a$ & $0.7$ & $0.7$\\
\midrule
autocorrelation($u$)& $0.90$& $0.91$\\
autocorrelation($v$)&$0.76$& $0.93$\\
\midrule
correlation($u,v$)&\al{$-0.89$} &\al{$-0.89$}\\
\bottomrule
\end{tabular*}
\end{table}}
\only<2>{\begin{itemize}
\item the volatility of unemployment and vacancies is as large in the model as in US data
\begin{itemize}
\item[\then] no Shimer [2005] puzzle
\item although wages are as flexible as in newly created US jobs
\end{itemize}
\item the correlation between unemployment and vacancies is the same in the model as in the data
\begin{itemize}
\item[\then] realistic Beveridge curve
\end{itemize}
\end{itemize}}
\end{frame}

\begin{frame}
\frametitle{historical decomposition of unemployment}
\begin{figure}
\includegraphics<1>[scale=\sfig,page=23]{\pdf}%
\includegraphics<2>[scale=\sfig,page=24]{\pdf}%
\includegraphics<3>[scale=\sfig,page=25]{\pdf}%
\includegraphics<4>[scale=\sfig,page=26]{\pdf}%
\includegraphics<5>[scale=\sfig,page=27]{\pdf}%
\includegraphics<6>[scale=\sfig,page=28]{\pdf}%
\includegraphics<7>[scale=\sfig,page=29]{\pdf}%
\end{figure}
\end{frame}

\begin{frame}
\frametitle{unemployment in model \& data}
\begin{figure}
\includegraphics[scale=\sfig,page=30]{\pdf}%
\end{figure}
\end{frame}


\begin{frame}
\heading{conclusion}
\end{frame}

\begin{frame}
\frametitle{summary}
\begin{itemize}
\item this paper develops a matching model with job rationing
\begin{itemize}
\item unemployment does not disappear when recruiting costs vanish
\end{itemize}
\item in booms: most of unemployment is frictional
\begin{itemize}
\item there are enough jobs
\item but the matching process and recruiting costs create unemployment
\end{itemize}
\end{itemize}
\end{frame}

\begin{frame}
\frametitle{summary}
\begin{itemize}
\item in slumps: frictional unemployment is lower and unemployment mostly comes from job rationing
\begin{itemize}
\item there are not enough jobs
\item the matching process and recruiting costs create little additional unemployment
\end{itemize}
\item simulations: 
\begin{itemize}
\item as unemployment \up from $4.8\%$ to $8.3\%$
\item rationing unemployment \up from $0\%$ to $7\%$
\item frictional unemployment \down from $4.8\%$ to $1.3\%$ 
\end{itemize}
\end{itemize}
\end{frame}

\begin{frame}
\frametitle{implications for modeling unemployment}
\begin{itemize}
\item the result that frictional unemployment is low in slumps does not mean that the matching framework is inappropriate to describe slumps
\item but it means that in slumps, the matching process and recruiting costs create little unemployment 
\item instead, most unemployment arises from a shortage of jobs---a weak labor demand
\end{itemize}
\end{frame}

\begin{frame}
\frametitle{implications for policy}
\begin{itemize}
\item in slumps: unemployment comes from job rationing 
\item[\then] to reduce unemployment in slumps, it is necessary to stimulate labor demand
\item[\then] policies reducing frictional unemployment have limited scope in slumps
\begin{itemize}
\item example \#1: creating a placement agency to improve matching
\item example \#2: reducing unemployment insurance to stimulate job search
\end{itemize}
\end{itemize}
\end{frame}


\begin{frame}
\frametitle{application \#1: unemployment insurance}
\begin{itemize}
\item the model can be combined with a Baily-Chetty model of optimal unemployment insurance (UI)
\item this model explains the rat-race effect: higher UI alleviates the rat race for jobs and raises tightness
\item policy implication: optimal UI is more generous in slumps than in booms
\item see \href{https://www.pascalmichaillat.org/4.html}{Landais, Michaillat, \& Saez [2018]}
\end{itemize}
\end{frame}


\begin{frame}
\frametitle{application \#2: countercyclical multipliers}
\begin{itemize}
\item the labor market model can be embedded into a New Keynesian model
\item this model explains the countercyclicality of the government multiplier
\item the result relies not on the zero lower bound but on the nonlinearity of the labor market
\item see \href{https://www.pascalmichaillat.org/2.html}{Michaillat [2014]}
\end{itemize}
\end{frame}

\begin{frame}
\frametitle{application \#3: unemployment fluctuations}
 \begin{itemize}
\item the labor market model can be combined to a product market model with a similar structure
\item this general-equilibrium model describes how unemployment fluctuations may arise from
\begin{itemize}
\item aggregate demand shocks
\item technology shocks
\item labor supply shocks
\end{itemize} 
\item in the US: most unemployment fluctuations come from aggregate demand shocks
\item see \href{https://www.pascalmichaillat.org/3.html}{Michaillat \& Saez [2015]}
\end{itemize}
\end{frame}

\end{document}
