\documentclass[12pt,xcolor={dvipsnames},hyperref={pdftex,pdfpagemode=UseNone,hidelinks,pdfdisplaydoctitle=true},usepdftitle=false]{beamer}
\usepackage{presentation,math}
\def\pdf{xstimulus.pdf}
\hypersetup{pdftitle={Lecture | Optimal Public Expenditure with Inefficient Unemployment}}

\begin{document}

\title{Optimal Public Expenditure with Inefficient Unemployment}
\information{Pascal Michaillat, Emmanuel Saez}%
{Review of Economic Studies, 2019}%
{https://www.pascalmichaillat.org/6.html}
\frame[plain]{\titlepage}

\begin{frame}
\frametitle{Main stabilization policy: monetary policy}
\begin{itemize}
\item policymakers rely primarily on monetary policy for stabilization
\begin{itemize}
\item accordingly: extensive research on optimal monetary policy
\end{itemize}
\item but monetary policy is sometimes constrained
\begin{itemize}
\item zero lower bound (Japan, USA, EU)
\item monetary union (EU, USA)
\item[\then] high unemployment
\end{itemize}
\item then other stabilization policies are needed
\begin{itemize}
\item but: very little is known about these alternative policies 
\end{itemize}
\end{itemize}
\end{frame}

\begin{frame}
\frametitle{This paper: optimal public expenditure} 
\begin{itemize}
\item public expenditure is commonly used for stabilization
\begin{itemize}
\item US: Great Depression (New Deal), Great Recession (ARRA)
\end{itemize}
\item framework: matching model from \href{https://www.pascalmichaillat.org/3.html}{Michaillat \& Saez (2015)}
\item outcome: formula linking optimal stimulus spending to 3 sufficient statistics
\begin{enumerate}
\item unemployment gap
\item unemployment multiplier
\item elasticity of substitution between public consumption \& private consumption
\end{enumerate}
\end{itemize}
\end{frame}

\begin{frame}
\frametitle{Optimal public expenditure: existing results}	
\begin{itemize}    
\item Samuelson (1954):
\begin{itemize}
	\item public goods financed by lump-sum taxation
	\item efficient level of production
	\item rule: spend until marginal utilities are equalized
	\item but: what if production is inefficient?
\end{itemize}
\item Keynes (1936):
\begin{itemize}
	\item no tradeoffs between public consumption \& private consumption (multiplier $>1$)
	\item rule: spend to fill output gap
	\item but: what if there is a tradeoff?
\end{itemize}
\item our theory blends the theories of Samuelson \& Keynes
\end{itemize}
\end{frame}

\begin{frame} 
\heading{Informal description of the model}
\end{frame}

\begin{frame}
\frametitle{A service economy, without firms}
\includegraphics<1>[scale=\sfig,page=1]{\pdf}%
\includegraphics<2>[scale=\sfig,page=2]{\pdf}%
\end{frame}

\begin{frame}
\frametitle{An asset for saving}
\includegraphics[scale=\sfig,page=3]{\pdf}%
\end{frame}

\begin{frame}
\frametitle{Private services ($c$) \& public services ($g$)}
\includegraphics<1>[scale=\sfig,page=4]{\pdf}%
\includegraphics<2>[scale=\sfig,page=5]{\pdf}%
\end{frame}

\begin{frame}
\frametitle{Matching: not all services are sold}
\includegraphics<1>[scale=\sfig,page=8]{\pdf}%
\includegraphics<2>[scale=\sfig,page=9]{\pdf}%
\end{frame}

\begin{frame}
\frametitle{Matching: costly to purchase services}
\includegraphics<1>[scale=\sfig,page=6]{\pdf}%
\includegraphics<2>[scale=\sfig,page=7]{\pdf}%
\end{frame}

\begin{frame}
\frametitle{Socially efficient rate of unemployment}
\begin{itemize}    
\item too much unemployment is bad
\begin{itemize}
	\item too many services are idle
\end{itemize}
\item too little unemployment is bad
\begin{itemize}
	\item too many services are devoted to recruiting
\end{itemize}
\item there is a socially efficient rate of unemployment ($u^*$)
\begin{itemize}
	\item number of services enjoyed ($y=g+c$) is maximized
\end{itemize}
\item[\then] when unemployment is efficient, Samuelson rule holds
\end{itemize}
\end{frame}

\begin{frame} 
\heading{Formal description of the model}
\end{frame}

\begin{frame}
\frametitle{Structure}
\begin{itemize}
\item dynamic matching model
\begin{itemize}
\item building on \href{https://www.pascalmichaillat.org/3.html}{Michaillat \& Saez (2015)}
\end{itemize}
\item identical, self-employed households
\item government
\item 2 consumption goods traded on a matching market
\begin{itemize}
\item public services \& private services
\end{itemize}
\item 1 asset for saving
\end{itemize}
\end{frame}

\begin{frame}
\frametitle{Matching market}
\begin{itemize}
\item capacity of each household: $k$ services
\item household purchases: $C(t)$ private services
\item government purchases: $G(t)$ public services
\item output: $Y(t)=C(t)+G(t)<k$
\item \al{unemployment rate: $u(t)=1-Y(t)/k$}
\item price of services: $p(t)$
\end{itemize}
\end{frame}

\begin{frame}
\frametitle{Matching function}
\begin{itemize}
\item number of vacancies: $v(t)$ 
\item matching function: $h(t)=\o\cdot \bs{k-Y(t)}^{\h}\cdot v(t)^{1-\h}$
\item \al{market tightness: $x(t)= v(t)/(k-Y(t))$}
\item selling rate \& buying rate: 
\begin{align*}
f(x(t))&=\frac{h(t)}{k-Y(t)} = \o\cdot x(t)^{1-\h}\\
q(x(t))&= \frac{h(t)}{v(t)}= \o\cdot x(t)^{-\h}
\end{align*}
\end{itemize}
\end{frame}

\begin{frame}
\frametitle{Market flows}
\begin{itemize}
\item relationships separate at rate $s$
\item given $x$, output and unemployment converge to
\begin{equation*}
Y(x,k) =\frac{f(x)}{s+f(x)}\cdot k,\quad u(x) =\frac{s}{s+f(x)}
\end{equation*}
\item convergence to steady state is extremely fast, so we assume:
\begin{itemize}
\item \al{$Y(t)=Y(\underset{+}{x(t)},k)$}
\item \al{$u(t)=u(\underset{-}{x(t)})$}
\item see Hall (2005)
\end{itemize}
\end{itemize}
\end{frame}

\begin{frame}
\frametitle{Matching cost: $\r$ services per vacancy}
\begin{itemize}
\item output ($Y$) = consumption ($y$) + matching cost
\begin{equation*}
Y  = y + \r\cdot v= y + s\cdot Y \cdot \frac{\r}{q(x)}
\end{equation*}
\item \al{matching wedge: $\tau(\underset{+}{x})=s\cdot\r/\bs{q(x)-s\cdot \r}$} 
\item total consumption: $y=Y/\bs{1+\tau(x)}$
\item private consumption: $c=C/\bs{1+\tau(x)}$
\item public consumption: $g=G/\bs{1+\tau(x)}$
\end{itemize}
\end{frame}

\begin{frame}
\frametitle{Supply structure: summary}
\begin{figure}
\includegraphics<1>[scale=\sfig,page=10]{\pdf}%
\includegraphics<2>[scale=\sfig,page=11]{\pdf}%
\includegraphics<3>[scale=\sfig,page=12]{\pdf}%
\includegraphics<4>[scale=\sfig,page=13]{\pdf}%
\includegraphics<5>[scale=\sfig,page=14]{\pdf}%
\end{figure}
\end{frame}

\begin{frame}
\frametitle{Demand structure: example}
\begin{itemize}
\item asset: land $l(t)$ in fixed supply $l_0$
\begin{itemize}
\item traded on a competitive market
\item Iacoviello (2005) and Liu, Wang, Zha (2013)
\end{itemize}
\item households choose $c(t)$ and $l(t)$ to maximize utility
\begin{equation*}
\int_{0}^{+\infty}e^{-\d\cdot t}\cdot \bs{\Uc(c,g)+ \al{\Vc\bp{l}}} dt
\end{equation*}
\item subject to flow budget constraint
\begin{equation*}
\dot{l}= p\cdot\al{\bs{1-u(x)}} \cdot k - p\cdot\al{\bs{1+\tau(x)}}\cdot c - T
\end{equation*}
\end{itemize}
\end{frame}

\begin{frame}
\frametitle{Aggregate demand in the example}
\begin{itemize}
\item market clearing on housing market: $l=l_0$
\item private demand $c^{d}(x,g,p)$ is solution to Euler equation:
\begin{equation*}
\pd{\Uc}{c}(\al{c},g)=\frac{p\cdot (1+\tau(x))\cdot \Vc'(l_0)}{\d}
\end{equation*}
\item price of services relative to housing: $p=p(x,g)$
\begin{itemize}
\item general price mechanism
\item (assumption required in matching model)
\end{itemize}
\end{itemize}
\end{frame}

\begin{frame}
\frametitle{Equilibrium tightness $x(g)$}
\begin{figure}
\includegraphics[scale=\sfig,page=15]{\pdf}%
\end{figure}
\end{frame}

\begin{frame}
\frametitle{Unemployment multiplier $m$}
\begin{figure}
\includegraphics[scale=\sfig,page=16]{\pdf}%
\end{figure}
\end{frame}

\begin{frame}
\frametitle{Socially efficient unemployment rate $u^*$}
\begin{figure}
\includegraphics[scale=\sfig,page=17]{\pdf}%
\end{figure}
\end{frame}

\begin{frame}
\frametitle{Inefficiently high unemployment rate}
\begin{figure}
\includegraphics[scale=\sfig,page=18]{\pdf}%
\end{figure}
\end{frame}

\begin{frame}
\frametitle{Inefficiently low unemployment rate}
\begin{figure}
\includegraphics[scale=\sfig,page=19]{\pdf}%
\end{figure}
\end{frame}

\begin{frame}
\heading{Optimal public expenditure}
\end{frame}

\begin{frame}
\frametitle{Government's problem}
\begin{itemize}
\item households' flow utility is $\Uc(c,g)$
\item public expenditure is financed by a lump-sum tax to maintain a balanced budget
\item given $x(g)$, the government chooses $g$ to maximize 
\begin{equation*}
\Uc\of{\underbrace{y(x(g),k)-g}_{c},g}
\end{equation*}
\end{itemize}
\end{frame}

\begin{frame}
\frametitle{Correcting the Samuelson formula}
\begin{itemize}
\item first-order condition of government's problem is
\begin{equation*}
0 =\pd{\Uc}{g}-\pd{\Uc}{c}+\pd{\Uc}{c}\cdot \pd{y}{x} \cdot \od{x}{g}
\end{equation*}
\item optimal public expenditure satisfies
\begin{equation*}
\underbrace{1 = MRS_{gc}}_{\text{Samuelson formula}}+ \al{\underbrace{\pd{y}{x} \cdot \od{x}{g}}_{\text{correction}}}
\end{equation*}
\begin{itemize}
\item $MRS_{gc} = (\pdx{\Uc}{g})/(\pdx{\Uc}{c})$ 	
\item correction due to effect of public expenditure on welfare through tightness
\end{itemize}
\end{itemize}
\end{frame}


\begin{frame}
\frametitle{Introducing estimable statistics}
\begin{itemize}
\item $(g/c)^{*}$: Samuelson spending
\item elasticity of substitution between $g$ and $c$:
\begin{equation*}
1-MRS_{gc}  \approx \frac{1}{\al{\e}}\cdot \frac{g/c-(g/c)^*}{(g/c)^*}
\end{equation*}
\item unemployment gap:
\begin{equation*}
\pd{y}{x} \propto \al{u-u^{*}}
\end{equation*}
\item unemployment multiplier:
\begin{equation*}
\od{x}{g} \propto \al{m} = -\frac{y}{1-u}\cdot\od{u}{g}
\end{equation*}
\end{itemize}
\end{frame}


\begin{frame}
\frametitle{Implicit formula for optimal stimulus}
\vspace{-1cm}
\begin{equation*}
\frac{g/c-(g/c)^{*}}{(g/c)^{*}} \approx z_0 \e m \cdot \frac{u-u^{*}}{u^{*}}
\end{equation*}
\vspace{-0.5cm}
\begin{itemize}
\item $g/c-(g/c)^{*}$: stimulus spending 
\item $\e$: elasticity of substitution between $g$ and $c$
\begin{itemize}
\item[=] marginal social value of public spending
\end{itemize}
\item $m$: unemployment multiplier
\begin{itemize}
	\item decrease in $u$ when $g$ increases by 1\% of $y$
\end{itemize}
\item $u-u^{*}$: unemployment gap
\begin{itemize}
\item[=] productive inefficiency 
\end{itemize}
\item $z_0$: constant of the parameters $\h$, $u^*$
\end{itemize}
\end{frame}


\begin{frame}
\frametitle{Departures from Samuelson rule}
\begin{table}
\begin{tabular*}{\textwidth}{@{\extracolsep\fill}lccc}
 & $m < 0$ & $m = 0$ & $m > 0$\\
\toprule
$u>u^{*}$  & $\ral{g/c<(g/c)^{*}}$  & $\bal{g/c=(g/c)^{*}}$ & $\gal{g/c>(g/c)^{*}}$ \\
$u=u^{*}$ & $\bal{g/c=(g/c)^{*}}$ & $\bal{g/c=(g/c)^{*}}$  & $\bal{g/c=(g/c)^{*}}$ \\ 
$u<u^{*}$ & $\gal{g/c>(g/c)^{*}}$ & $\bal{g/c=(g/c)^{*}}$ & $\ral{g/c<(g/c)^{*}}$ \\ 
\bottomrule
\end{tabular*}
\end{table}
\end{frame}

\begin{frame}
\frametitle{Marginal value of public services}
\begin{itemize}
\item $\e=0$: digging holes or building pyramids
\begin{itemize}
\item $g/c=(g/c)^{*}$: Samuelson rule holds, no stimulus spending
\end{itemize}
\item $\e\to +\infty$: perfect substitution
\begin{itemize}
\item $u=u^{*}$: entirely fill unemployment gap, as in Keynes
\end{itemize}
\item $\e\in (0,+\infty)$: medium substitution
\begin{itemize}
\item medium stabilization: $g/c\neq (g/c)^{*}$ but $u\neq u^{*}$
\item[\then] partially fill unemployment gap
\end{itemize}
\end{itemize}
\end{frame}

\begin{frame}
\frametitle{Making the formula explicit}
\begin{itemize}
\item implicit formula: not useful for quantitative results because $u$ in RHS responds to $g/c$ in LHS
\item starting from $(g/c)^{*}$ and $u_{0}\neq u^{*}$:
\begin{equation*}
\frac{g/c-(g/c)^{*}}{(g/c)^{*}}\approx z_0 \e m\cdot \frac{\al{u(g/c)}-u^{*}}{u^{*}}
\end{equation*}
\item first-order Taylor expansion of $u$ at $u((g/c)^{*})=u_{0}$:
\begin{equation*}
\frac{u-u^{*}}{u^{*}}\approx \frac{\al{u_{0}}-u^{*}}{u^{*}}-z_1 \al{m} \cdot\frac{g/c-(g/c)^{*}}{(g/c)^{*}}
\end{equation*}
\item $z_1$: constant of the parameters $u^*$, $(g/c)^*$
\end{itemize}
\end{frame}

\begin{frame}
\frametitle{Explicit formula}
\begin{itemize}
\item optimal $g/c$ depends on fixed quantities:
\begin{equation*}
\frac{g/c-(g/c)^{*}}{(g/c)^{*}}\approx \frac{z_0\e m}{1+  z_1 z_0 \e m^{2}} \cdot \frac{\al{u_{0}}-u^{*}}{u^{*}}
\end{equation*}
\item optimal $u$ depends on fixed quantities:
\begin{equation*}
u\approx u^{*}+\frac{u_{0}-u^{*}}{1+ z_1 z_0 \e m^{2}}
\end{equation*}
\item approximations valid up to 2nd-order terms
\end{itemize}
\end{frame}


\begin{frame}
\frametitle{Results with distortionary taxation}
\begin{itemize}
\item endogenous capacity: $\Uc(c,g,k)$ with $\pdx{\Uc}{k}<0$
\item linear income tax: $T=\al{\tau^{L}}\cdot (1-u(x))\cdot k$
\item everything remains valid
\begin{itemize}
 \item but $(g/c)^{*}$ is lower because of tax distortions
 \end{itemize} 
\item however: link between multipliers changes
\begin{itemize}
\item no tax  distortions: $ m = dY/dG$
\item tax distortions:  $ m > dY/dG$
\item with taxes, we may have $dY/dG<0$ but $m>0$
\end{itemize}
\end{itemize}
\end{frame}


\begin{frame}
\heading{Numerical illustration:\\Great Recession in the US}
\end{frame}

\begin{frame}
\frametitle{Starting point: winter 2008--2009}
\begin{itemize}
\item unemployment $=6\%$ and public spending $=16.5\%$ of GDP
\begin{itemize}
	\item for illustration: we take these values as efficient
\end{itemize}
\item unemployment is forecast to increase to $9\%$
\begin{itemize}
	\item initial unemployment gap $= 9\% - 6\% = 3\%$
\end{itemize}
\item we compute optimal stimulus for various elasticities of substitution and unemployment multipliers
\end{itemize}
\end{frame}

\begin{frame}
\frametitle{Optimal stimulus spending (\% of GDP)}
\includegraphics<1>[scale=\sfig,page=20]{\pdf}%
\includegraphics<2>[scale=\sfig,page=21]{\pdf}%
\includegraphics<3>[scale=\sfig,page=22]{\pdf}%
\includegraphics<4>[scale=\sfig,page=23]{\pdf}%
\includegraphics<5>[scale=\sfig,page=24]{\pdf}%
\end{frame}

\begin{frame}
\frametitle{Optimal stimulus spending for various $\e$}
\includegraphics[scale=\sfig,page=25]{\pdf}%
\end{frame}

\begin{frame}
\frametitle{Unemployment under optimal stimulus}
\includegraphics[scale=\sfig,page=26]{\pdf}%
\end{frame}

\begin{frame}
\heading{Some simulations}
\end{frame}

\begin{frame}
\frametitle{Optimal stimulus in calibrated model}
\includegraphics[scale=\sfig,page=27]{\pdf}%
\end{frame}

\begin{frame}
\frametitle{Unemployment rate in calibrated model}
\includegraphics[scale=\sfig,page=28]{\pdf}%

\end{frame}

\begin{frame}
\frametitle{Multiplier in calibrated model}
\includegraphics[scale=\sfig,page=29]{\pdf}%
\end{frame}

\begin{frame}
\frametitle{Quality of approximations in formula}
\includegraphics[scale=\sfig,page=30]{\pdf}%
\end{frame}

\begin{frame}
\heading{Summary \& discussion}
\end{frame}

\begin{frame}
\begin{enumerate}
\item  $dY/dG>1$ is not necessary for stimulus
\begin{itemize}
\item stimulus requires unemployment multiplier $>0$ (as in data)
\end{itemize}
\item bang-for-the-buck logic does not hold
\begin{itemize}
\item strongest stimulus for $m=0.4$
\item same stimulus for $m=0.1$ and $m=1.4$
\end{itemize}
\item completely filling the unemployment gap is not optimal
\begin{itemize}
\item optimal to partially fill unemployment gap 
\item except if public services = private services
\end{itemize}
\item low marginal social value of $g$ does not imply no stimulus
\begin{itemize}
\item optimal to reduce unemployment gap 
\item except if public services = digging holes
\end{itemize}
\end{enumerate}
\end{frame}

\begin{frame}
\frametitle{Distortionary taxes $\nRightarrow$ smaller stimulus}
\begin{itemize}
\item formula remains valid with distortionary taxation
\begin{itemize}
	\item but Samuelson spending is lower
\end{itemize}
\item however, $dY/dG$ is not useful anymore because $dY/dG \neq m$
\begin{itemize}
\item $dY/dG = m$ + labor-supply response to taxes 
\item labor-supply distortion reduces $dY/dG$ but not $m$
\item so: $ m > dY/dG$
\item possibly: $dY/dG<0$ while $m>0$
\end{itemize}
\item distortionary taxation does not imply smaller stimulus
\begin{itemize}
\item only average public spending is lower
\end{itemize}
\end{itemize}
\end{frame}


\end{document}
