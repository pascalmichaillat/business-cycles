\documentclass[12pt,xcolor={dvipsnames},hyperref={pdftex,pdfpagemode=UseNone,hidelinks,pdfdisplaydoctitle=true},usepdftitle=false]{beamer}
\usepackage{presentation,math}
\def\pdf{xgap.pdf}
\hypersetup{pdftitle={Lecture | Beveridgean Unemployment Gap}}

\def\W{\wh{\Wc}}

\begin{document}

\title{Beveridgean Unemployment Gap}
\information{Pascal Michaillat, Emmanuel Saez}%
{Journal of Public Economics Plus, 2021}%
{https://www.pascalmichaillat.org/9.html}
\frame[plain]{\titlepage}

\begin{frame}
\frametitle{Does the labor market operate efficiently?}
\begin{itemize}
\item we develop welfare-based measure of unemployment gap 
\begin{itemize}
\item[=] actual unemployment rate $-$ efficient unemployment rate
\end{itemize}
\item[\then] model design
\begin{itemize}
\item bargained wages or competitive search?
\item rigid wages?
\end{itemize} 
\item[\then] distance from ``full employment''
\item[\then] optimal macro policies
\begin{itemize}
\item monetary policy
\item fiscal policy
\item unemployment insurance
\end{itemize}
\end{itemize}
\end{frame}

\begin{frame}
\heading{Theory}
\end{frame}

\begin{frame}
\frametitle{US Beveridge curve}
\includegraphics<1>[scale=\sfig,page=1]{\pdf}%
\includegraphics<2>[scale=\sfig,page=2]{\pdf}%
\end{frame}

\begin{frame}
\frametitle{Condition for labor-market efficiency}
\includegraphics<1>[scale=\sfig,page=3]{\pdf}%
\includegraphics<2>[scale=\sfig,page=4]{\pdf}%
\includegraphics<3>[scale=\sfig,page=5]{\pdf}%
\includegraphics<4>[scale=\sfig,page=6]{\pdf}%
\end{frame}

\begin{frame}
\frametitle{Unemployment gap}
\includegraphics<1>[scale=\sfig,page=7]{\pdf}%
\includegraphics<2>[scale=\sfig,page=8]{\pdf}%
\end{frame}

\begin{frame}
\frametitle{Beveridgean model of labor market}
\begin{enumerate}
\item Beveridge curve: $v(u)$
\begin{itemize}
\item $v$: vacancy rate
\item $u$: unemployment rate
\item $v(u)$: decreasing in $u$, convex
\end{itemize}
\item social welfare: $\W(u,v) = \Wc(n,u,v)$ with $n=1-u$
\begin{itemize}
\item $n$: employment rate
\item $\Wc$: production + recruiting + preferences
\item $\W(u,v)$: decreasing in $u$ and $v$, quasiconcave  
\end{itemize}
\end{enumerate}
\end{frame}

\begin{frame}
\frametitle{Graphical condition for efficiency}
\begin{itemize}
\item efficiency at tangency point: $v'(u) = MRS_{uv}$
\item decomposing the social marginal rate of substitution:
\only<1>{\begin{equation*}
MRS_{uv} = -\frac{\pdx{\W}{u}}{\pdx{\W}{v}}
\end{equation*}}%
\only<2>{\begin{equation*}
MRS_{uv} = -\frac{\pdx{\Wc}{u}-\pdx{\Wc}{n}}{\pdx{\Wc}{v}} 
\end{equation*}}%
\only<3-4>{\begin{equation*}
MRS_{uv} = -\frac{1 - (\pdx{\Wc}{u})/(\pdx{\Wc}{n})}{-(\pdx{\Wc}{v})/(\pdx{\Wc}{n})} 
\end{equation*}}%
\item<4> social value of nonwork: $\z = (\pdx{\Wc}{u})/(\pdx{\Wc}{n})<1$
\item<4> recruiting cost: $\k = - (\pdx{\Wc}{v})/(\pdx{\Wc}{n}) > 0$
\item<4> efficiency condition:
\begin{equation*}
v'(u)  = - \frac{1 - \z}{\k} 
\end{equation*}
\end{itemize}
\end{frame}


\begin{frame}
\frametitle{Sufficient-statistic formula for efficiency}
\begin{itemize}
\item labor market tightness: $\t = v/u$	
\item Beveridge elasticity: $\e = -\odlx{v}{u}>0$
\item efficient labor market tightness:
\only<1>{\begin{equation*}
v'(u) = -\frac{1 - \z}{\k}
\end{equation*}}%
\only<2>{\begin{equation*}
-\frac{v'(u)}{v/u} \cdot\frac{v}{u}  =  \frac{1 - \z}{\k}
\end{equation*}}%
\only<3>{\begin{equation*}
\t  =  \frac{1 - \z}{\k \cdot \e}
\end{equation*}}%
\only<4-6>{\begin{equation*}
\al{\t^*  =  \frac{1 - \z}{\k \cdot \e }}
\end{equation*}}%
\item<5-6> $u^*$ obtained from $\t^*$ through Beveridge curve
\only<5>{\begin{equation*}
\frac{u^*}{u} = \bp{\frac{\t^*}{\t}}^{-1/(1+\e)}
\end{equation*}}%
\only<6>{\begin{equation*}
\al{u^* = \bp{\frac{\k \cdot \e}{1-\z}\cdot\frac{v}{u^{-\e}}}^{1/(1+\e)}}
\end{equation*}}%
\end{itemize}
\end{frame}

\begin{frame}
\heading{Application to the United States}
\end{frame}

\begin{frame}
\frametitle{Unemployment rate (CPS)}
\includegraphics[scale=\sfig,page=9]{\pdf}%
\end{frame}

\begin{frame}
\frametitle{Vacancy rate (Barnichon 2010 \& JOLTS)}
\includegraphics[scale=\sfig,page=10]{\pdf}%
\end{frame}

\begin{frame}
\frametitle{Beveridge-curve branches (Bai, Perron 1998)}
\includegraphics<1>[scale=\sfig,page=11]{\pdf}%
\includegraphics<2>[scale=\sfig,page=12]{\pdf}%
\includegraphics<3>[scale=\sfig,page=13]{\pdf}%
\includegraphics<4>[scale=\sfig,page=14]{\pdf}%
\includegraphics<5>[scale=\sfig,page=15]{\pdf}%
\includegraphics<6>[scale=\sfig,page=16]{\pdf}%
\end{frame}

\begin{frame}
\frametitle{Beveridge elasticity (Bai, Perron 1998) }
\includegraphics[scale=\sfig,page=17]{\pdf}%
\end{frame}

\begin{frame}
\frametitle{Social value of nonwork}
\begin{itemize}
\item Borgschulte, Martorell (2018): natural experiment using military administrative data
\begin{itemize}
\item 420,000 veterans
\item home production + recreation = $13\%$--$35\%$ earnings
\end{itemize}
\item Mas, Pallais (2019): field experiment in which job applicants choose wage-hour bundles
\begin{itemize}
\item 900 subjects
\item home production + recreation = $58\%$ earnings
\end{itemize}
\item[\then] \al{$\z\in [0.03,0.49]$, with median value of $\z=0.26$}
\end{itemize}
\end{frame}

\begin{frame}
\frametitle{Recruiting cost}
\begin{itemize}
\item 1997 National Employer Survey, administered by Census Bureau
\begin{itemize}
	\item 2,000 establishments
	\item establishments have $\geq 20$ workers 
	\item establishments belong to all industries
\end{itemize}
\item recruiting $= 3.2\%$ of labor costs 
\item[\then] \al{$\k = 0.92$}
\end{itemize}
\end{frame}

\begin{frame}
\frametitle{Efficient tightness \& tightness gap}
\includegraphics[scale=\sfig,page=18]{\pdf}
\end{frame}

\begin{frame}
\frametitle{Efficient unemployment \& unemployment gap}
\includegraphics[scale=\sfig,page=19]{\pdf}
\end{frame}

\begin{frame}
\frametitle{Comparison with existing ``natural rates''}
\includegraphics[scale=\sfig,page=20]{\pdf}
\end{frame}

\begin{frame}
\heading{Alternative calibrations of statistics}
\end{frame}

\begin{frame}
\frametitle{Beveridge elasticity in 95\% CI}
\includegraphics[scale=\sfig,page=21]{\pdf}%
\end{frame}

\begin{frame}
\frametitle{Inverse-optimum $\e$, so $u = u^*$}
\includegraphics[scale=\sfig,page=24]{\pdf}
\end{frame}

\begin{frame}
\frametitle{Plausible social values of nonwork}
\includegraphics[scale=\sfig,page=22]{\pdf}
\end{frame}

\begin{frame}
\frametitle{Inverse-optimum $\z$, so $u = u^*$}
\includegraphics[scale=\sfig,page=25]{\pdf}
\end{frame}

\begin{frame}
\frametitle{Plausible recruiting costs}
\includegraphics[scale=\sfig,page=23]{\pdf}
\end{frame}

\begin{frame}
\frametitle{Inverse-optimum $\k$, so $u = u^*$}
\includegraphics[scale=\sfig,page=26]{\pdf}
\end{frame}

\begin{frame}
\frametitle{Hagedorn, Manovskii (2008): $\z=0.96$}
\includegraphics[scale=\sfig,page=27]{\pdf}
\end{frame}

\begin{frame}
\heading{Application to Diamond-Mortensen-Pissarides model}
\end{frame}

\begin{frame}
\frametitle{Unemployment: on DMP Beveridge curve}
\includegraphics<1>[scale=\sfig,page=28]{\pdf}%
\includegraphics<2>[scale=\sfig,page=29]{\pdf}%
\end{frame}

\begin{frame}
\frametitle{Sufficient statistics in DMP model}
\begin{itemize}
\item Beveridge curve: UE flows $=$ EU flows
\begin{equation*}
v(u) = \bs{\frac{\l \cdot (1-u)}{\o \cdot u^{\h}}}^{1/(1-\h)}
\end{equation*}
\item[\then] Beveridge elasticity:
\begin{equation*}
\e = \frac{1}{1-\h}\bs{\h+ \frac{u}{1-u}}
\end{equation*}
\item social welfare: $\Wc(n,u,v) = p \cdot \bp{n  + z \cdot u - c \cdot v}$
\item[\then] social value of nonwork: $\z = z$
\item[\then] recruiting cost: $\k = c$
\end{itemize}
\end{frame}

\begin{frame}
\frametitle{DMP business cycles in Beveridge diagram}
\includegraphics<1>[scale=\sfig,page=30]{\pdf}%
\includegraphics<2>[scale=\sfig,page=31]{\pdf}%
\includegraphics<3>[scale=\sfig,page=32]{\pdf}%
\end{frame}

\begin{frame}
\frametitle{Beveridgean efficiency $\approx$ Hosiosian efficiency}
\includegraphics<1>[scale=\sfig,page=33]{\pdf}%
\includegraphics<2>[scale=\sfig,page=34]{\pdf}%
\end{frame}

\begin{frame}
\heading{Conclusion}
\end{frame}

\begin{frame}
\frametitle{Summary}
\begin{itemize}
\item socially efficient unemployment rate $u^*$ \& unemployment gap $u-u^*$ are determined by 3 sufficient statistics
\begin{itemize}
\item elasticity of Beveridge curve
\item social cost of unemployment
\item cost of recruiting
\end{itemize} 
\item in the United States, 1951--2019:
\begin{itemize}
\item $u^*$ averages $4.3\%$ \then $u-u^*$ averages $1.4$pp
\item $3.0\% < u^*  < 5.4\%$ \then $u-u^*$ is countercyclical
\item[\then] \al{labor market is inefficient}
\item[\then] \al{labor market is inefficiently slack in slumps}
\end{itemize}
\end{itemize}
\end{frame}

\begin{frame}
\frametitle{Implications for model design}
\begin{itemize}
\item models featuring an \al{efficient labor market} are \al{inconsistent} with our findings
\begin{itemize}
\item DMP model with Hosios (1990) condition
\item models with competitive-search equilibrium (Moen 1997)
\end{itemize}
\item models producing a \al{countercyclical unemployment gap} are \al{consistent} with our findings
\begin{itemize}
\item DMP model with bargaining-power shocks (Shimer 2005)
\item variant of the DMP model with rigid wages (Hall 2005)
\end{itemize} 
\end{itemize}
\end{frame}

\begin{frame}
\frametitle{Implications for policy design }
\begin{itemize}
\item optimal \al{nominal interest rate} is \al{procyclical}
\begin{itemize}
\item optimal for monetary policy to eliminate the unemployment gap \href{https://www.pascalmichaillat.org/7.html}{(Michaillat, Saez 2021)}
\item unemployment \up when interest rate \up (Coibion 2012)
\end{itemize}
\item optimal \al{government spending} is \al{countercyclical}
\begin{itemize}
\item optimal for government spending to reduce---but not eliminate---the unemployment gap \href{https://www.pascalmichaillat.org/6.html}{(Michaillat, Saez 2019)}
\item unemployment \down when spending \up (Ramey 2013)
\end{itemize}
\end{itemize}	
\end{frame}

\begin{frame}
\frametitle{Implications for policy design}
\begin{itemize}
\item optimal \al{unemployment insurance} is \al{countercyclical}
\begin{itemize}
\item US tightness gap is procyclical	
\item optimal for unemployment insurance to reduce the tightness gap \href{https://www.pascalmichaillat.org/4.html}{ (Landais, Michaillat, Saez 2018)}
\item tightness \up when unemployment insurance \up \href{https://www.pascalmichaillat.org/5.html}{ (Landais, Michaillat, Saez 2018)}
\end{itemize}
\end{itemize}	
\end{frame}

\end{document}
