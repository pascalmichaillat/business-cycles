\documentclass[12pt,xcolor={dvipsnames},hyperref={pdftex,pdfpagemode=UseNone,hidelinks,pdfdisplaydoctitle=true},usepdftitle=false]{beamer}
\usepackage{presentation,math}
\def\pdf{xwunk.pdf}
\hypersetup{pdftitle={Lecture | Resolving New Keynesian Anomalies With Wealth in the Utility Function}}

\begin{document}

\title{Resolving New Keynesian Anomalies With Wealth in the Utility Function}
\information{Pascal Michaillat, Emmanuel Saez}%
{Review of Economics and Statistics, 2021}%
{https://www.pascalmichaillat.org/11.html}
\frame[plain]{\titlepage}

\begin{frame}
\frametitle{anomalies in NK model at ZLB}
\begin{enumerate}
\item collapse of output \& inflation
\begin{itemize}
\item Eggertsson, Woodford [2004]
\item Werning [2011]
\end{itemize}
\item implausibly large effects of forward guidance
\begin{itemize}
\item del Negro, Giannoni, Patterson [2015]
\item Cochrane [2017]
\end{itemize}
\item implausibly large effects of government spending
\begin{itemize}
\item Christiano, Eichenbaum, Rebelo [2011]
\item Cochrane [2017]
\end{itemize}
\end{enumerate}
\end{frame}

\begin{frame}
\frametitle{existing remedies to ZLB anomalies}
\begin{itemize}
	\item Cochrane [2018]: fiscal theory of price level
	\item Bilbiie [2018] \& Acharya, Dogra [2020]: heterogeneous agents
  \item Gabaix [2020]: bounded rationality
	\item Diba, Loisel [2021]: interest on bank reserves
	\item but these remedies complicate the textbook model
	\begin{itemize}
	\item sometimes equilibrium system becomes 3-dimensional
	\item sometimes derivations are complicated by heterogeneity or bounded rationality
	\end{itemize}
\end{itemize}
\end{frame}

\begin{frame}
\frametitle{this paper: minimal deviation from textbook}
\begin{itemize}
\item New Keynesian model with relative wealth in the utility function
\item only one additional parameter
\begin{itemize}
\item marginal utility of wealth in Euler equation
\end{itemize}
\item[\then] equilibrium system remains 2-dimensional
\begin{itemize}
\item 2 variables: output \& inflation
\item 2 differential equations: Euler equation \& Phillips curve
\end{itemize}
\item[\then] derivations remain exactly the same
\end{itemize}
\end{frame}

\begin{frame}
\frametitle{why would people value wealth in itself?}
\begin{itemize}
\item Keynes [1919]: ``The duty of saving became nine-tenths of virtue and the growth of the cake the object of true religion\ldots. Saving was for old age or for your children; but \al{this was only in theory}---the virtue of the cake was that it was never to be consumed, neither by you nor by your children after you.''
\item Irving Fisher [1930]: ``A man may include in the benefits of his wealth\ldots the \al{social standing} he thinks it gives him, or political power and influence, or the mere miserly sense of possession, or the satisfaction in the mere process of further accumulation.'' 
\end{itemize}
\end{frame}

\begin{frame}
\frametitle{why would people value wealth in itself?}
\begin{itemize}
\item Camerer, Loewenstein, Prelec [2005]: ``brain-scans conducted while people win or lose money suggest that money activates similar reward areas as do other \al{primary reinforcers} like food and drugs, which implies that money confers direct utility, rather than simply being valued only for what it can buy.''
\item evidence from economics, social psychology, sociology, social neuroscience: wealth is a \al{marker of social status}, and people value high social status
\end{itemize}
\end{frame}


\begin{frame}
\heading{NK model with wealth in the utility}
\end{frame}

\begin{frame}
\begin{itemize}
\item self-employed household $j \in [0,1]$ maximizes utility
\begin{equation*}
\int_{0}^{\infty}e^{-\d t} \bs{\ln{c_{j}(t)}+\al{u\of{\frac{b_{j}(t)}{p(t)}-\frac{b(t)}{p(t)}}}- \k h_{j}(t)-\frac{\g}{2} \pi_{j}(t)^2}dt
\end{equation*}
\vspace{-0.5cm}
\begin{itemize}
\item consumption index: $c_{j}(t) = \bs{\int_{0}^{1}c_{jk}(t)^{(\e-1)/\e}\,dk}^{\e/(\e-1)}$
\item aggregate wealth:  $b(t) = \int_{0}^{1} b_{j}(t)\,dj$
\item inflation: $\pi_{j}(t) = \dot{p}_{j}(t)/p_{j}(t)$
\end{itemize}
\item  subject to budget constraint:
\begin{equation*}
\dot{b}_{j}(t) = i(t) b_{j}(t) + p_{j}(t)  y_{j}(t) - \int_0^1 p_{k}(t) c_{jk}(t)\,dk
\end{equation*}
\item to production function: $y_{j}(t) = a h_{j}(t)$
\item to demand for good $i$: $y_{j}(t) = \bs{p_{j}(t)/p(t)}^{-\e}  c(t) $ 
\end{itemize}
\end{frame}

\begin{frame}
\frametitle{equilibrium: euler-phillips system}
\begin{itemize}
\item Phillips curve: standard
\begin{equation*}
\dot{\pi} = \d\pi- \frac{\e\k}{\g a} \bp{y - y^n} \quad \text{with} \quad y^n = \frac{\e-1}{\e} \cdot \frac{a}{\k}
\end{equation*}
\item Euler equation: ``discounted''
\begin{equation*}
\frac{\dot{y}}{y} = r(\pi) + \al{u'(0) y} -\d
\end{equation*}
\vspace{-1cm}
\begin{itemize}
\item financial returns: real interest rate $= r(\pi) = i(\pi)-\pi$
\item hedonic returns: MRS(wealth, consumption) $= u'(0)y^n$ 
\end{itemize}
\end{itemize}
\begin{equation*}
\text{so} \quad \frac{\dot{y}}{y} = r(\pi)- r^n + \al{u'(0)} (y-y^n)\quad \text{with} \quad r^n =\d-u'(0)y^n
\end{equation*}

\end{frame}

\begin{frame}
\frametitle{two models}
\begin{itemize}
\item NK: standard New Keynesian model 
\begin{equation*}
u'(0)=0
\end{equation*}
\item WUNK: wealth-in-the-utility New Keynesian model
\begin{equation*}
u'(0)>\frac{\e\k}{\d\g a}
\end{equation*}
\end{itemize}
\end{frame}

\begin{frame}
\heading{output \& inflation collapse}
\end{frame}

\begin{frame}
\frametitle{scenario: ZLB}
\includegraphics<1>[scale=\sfig,page=1]{\pdf}%
\end{frame}

\begin{frame}
\frametitle{NK | phase diagram in normal times\only<5>{: source}}
\includegraphics<1>[scale=\sfig,page=2]{\pdf}%
\includegraphics<2>[scale=\sfig,page=3]{\pdf}%
\includegraphics<3>[scale=\sfig,page=4]{\pdf}%
\includegraphics<4>[scale=\sfig,page=5]{\pdf}%
\includegraphics<5>[scale=\sfig,page=6]{\pdf}%
\end{frame}

\begin{frame}
\frametitle{NK | phase diagram at ZLB\only<3>{: saddle}}
\includegraphics<1>[scale=\sfig,page=5]{\pdf}%
\includegraphics<2>[scale=\sfig,page=7]{\pdf}%
\includegraphics<3>[scale=\sfig,page=8]{\pdf}%
\end{frame}

\begin{frame}
\frametitle{NK | ZLB episode}
\includegraphics<1>[scale=\sfig,page=9]{\pdf}%
\includegraphics<2>[scale=\sfig,page=10]{\pdf}%
\end{frame}

\begin{frame}
\frametitle{NK | longer ZLB: output \& inflation collapse}
\includegraphics[scale=\sfig,page=11]{\pdf}%
\end{frame}

\begin{frame}
\frametitle{WUNK | phase diagram in normal times\only<3>{: source}}
\includegraphics<1>[scale=\sfig,page=5]{\pdf}%
\includegraphics<2>[scale=\sfig,page=12]{\pdf}%
\includegraphics<3>[scale=\sfig,page=13]{\pdf}%
\end{frame}

\begin{frame}
\frametitle{WUNK | phase diagram at ZLB\only<4>{: source}}
\includegraphics<1>[scale=\sfig,page=14]{\pdf}%
\includegraphics<2>[scale=\sfig,page=15]{\pdf}%
\includegraphics<3>[scale=\sfig,page=16]{\pdf}%
\includegraphics<4>[scale=\sfig,page=17]{\pdf}%
\end{frame}

\begin{frame}
\frametitle{WUNK | ZLB episode}
\includegraphics<1>[scale=\sfig,page=18]{\pdf}%
\includegraphics<2>[scale=\sfig,page=19]{\pdf}%
\end{frame}

\begin{frame}
\frametitle{WUNK | longer ZLB converges to steady state}
\includegraphics<1>[scale=\sfig,page=20]{\pdf}%
\end{frame}

\begin{frame}
\heading{forward guidance}
\end{frame}

\begin{frame}
\frametitle{scenario: ZLB + forward guidance}
\includegraphics<1>[scale=\sfig,page=21]{\pdf}%
\end{frame}


\begin{frame}
\frametitle{NK | \textbf<3->{ZLB} + \textbf<1-2>{forward guidance}}
\includegraphics<1>[scale=\sfig,page=22]{\pdf}%
\includegraphics<2>[scale=\sfig,page=23]{\pdf}%
\includegraphics<3>[scale=\sfig,page=24]{\pdf}%
\includegraphics<4>[scale=\sfig,page=25]{\pdf}%
\end{frame}

\begin{frame}
\frametitle{NK | longer guidance: boom at ZLB}
\includegraphics<1>[scale=\sfig,page=26]{\pdf}%
\includegraphics<2>[scale=\sfig,page=27]{\pdf}%
\includegraphics<3>[scale=\sfig,page=28]{\pdf}%
\end{frame}

\begin{frame}
\frametitle{WUNK | \textbf<3->{ZLB} + \textbf<1-2>{forward guidance}}
\includegraphics<1>[scale=\sfig,page=29]{\pdf}%
\includegraphics<2>[scale=\sfig,page=30]{\pdf}%
\includegraphics<3>[scale=\sfig,page=31]{\pdf}%
\includegraphics<4>[scale=\sfig,page=32]{\pdf}%
\end{frame}

\begin{frame}
\frametitle{WUNK | longer guidance: limited effect}
\includegraphics<1>[scale=\sfig,page=33]{\pdf}%
\end{frame}

\begin{frame}
\heading{Government spending}
\end{frame}

\begin{frame}
\frametitle{scenario: ZLB + government spending $g$}
\includegraphics[scale=\sfig,page=34]{\pdf}%
\end{frame}


\begin{frame}
\frametitle{NK | ZLB + no spending}
\includegraphics[scale=\sfig,page=35]{\pdf}%
\end{frame}


\begin{frame}
\frametitle{NK | ZLB + small spending}
\includegraphics[scale=\sfig,page=36]{\pdf}%
\end{frame}

\begin{frame}
\frametitle{NK | ZLB + medium spending}
\includegraphics[scale=\sfig,page=37]{\pdf}%
\end{frame}

\begin{frame}
\frametitle{NK | ZLB + large spending: boom at ZLB}
\includegraphics[scale=\sfig,page=38]{\pdf}%
\end{frame}

\begin{frame}
\frametitle{WUNK | ZLB + no spending}
\includegraphics[scale=\sfig,page=39]{\pdf}%
\end{frame}


\begin{frame}
\frametitle{WUNK | ZLB + small spending}
\includegraphics[scale=\sfig,page=40]{\pdf}%
\end{frame}

\begin{frame}
\frametitle{WUNK | ZLB + medium spending}
\includegraphics[scale=\sfig,page=41]{\pdf}%
\end{frame}

\begin{frame}
\frametitle{WUNK | ZLB + large spending: limited effect}
\includegraphics[scale=\sfig,page=42]{\pdf}%
\end{frame}


\begin{frame}
\heading{other ZLB properties in WUNK}
\end{frame}

\begin{frame}
\frametitle{paradox of thrift: higher MU of wealth}
\includegraphics<1>[scale=\sfig,page=43]{\pdf}%
\includegraphics<2>[scale=\sfig,page=44]{\pdf}%
\end{frame}

\begin{frame}
\frametitle{paradox of toil: lower disutility of labor}
\includegraphics<1>[scale=\sfig,page=45]{\pdf}%
\includegraphics<2>[scale=\sfig,page=46]{\pdf}%
\end{frame}

\begin{frame}
\frametitle{paradox of flexibility: lower price-adjustment cost}
\includegraphics<1>[scale=\sfig,page=47]{\pdf}%
\includegraphics<2>[scale=\sfig,page=48]{\pdf}%
\end{frame}

\begin{frame}
\frametitle{above-one government-spending multiplier}
\includegraphics<1>[scale=\sfig,page=49]{\pdf}%
\includegraphics<2>[scale=\sfig,page=50]{\pdf}%
\end{frame}

\begin{frame}
\heading{assessment of WUNK assumption}
\end{frame}

\begin{frame}
\begin{itemize}
\item WUNK assumption in measurable statistics:
\begin{equation*}
\d-r^n > \frac{\l}{\d}
\end{equation*}
\item $\d =$ annual time discount rate $\approx  43\%$
\begin{itemize}
\item Frederick, Loewenstein, O'Donoghue [2002]
\item Andersen, Harrison, Lau, Rutstrom [2014]
\end{itemize}
\item $r^n =$ natural rate of interest $\approx 2\%$
\item $\l= $ output-gap coefficient in Phillips curve $\approx 1.6\%$
\begin{itemize}
\item Mavroeidis, Plagborg-Moller, Stock [2014]
\end{itemize}
\item assumption holds: \al{$43\% - 2\% = 0.41>0.037 = 1.6\%/43\%$}
\begin{itemize}
\item lowest acceptable household discount rate: $27\%$
\item lowest acceptable firm discount rate: $16\%$
\end{itemize}
\end{itemize}
\end{frame}	

\end{document}
